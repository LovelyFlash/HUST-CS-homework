\section{第一章
数论初步及应用}\label{ux7b2cux4e00ux7ae0-ux6570ux8bbaux521dux6b65ux53caux5e94ux7528}

\subsection{1.1
除法和模运算}\label{ux9664ux6cd5ux548cux6a21ux8fd0ux7b97}

\subsubsection{1.1.1 整除}\label{ux6574ux9664}

\textbf{定义}:\(如果 a 和 b 是整数,且 a≠0,那么如果存在一个整数 c,使得 b=ac,则称 a 整除 b.\)
- 当 \(a\) 整除 \(b\) 时,我们称 \(a\) 是 \(b\) 的因数或除数,并且称
\(b\) 是 \(a\) 的倍数 - 符号 \(a \mid  b\) 表示 \(a\) 整除 \(b\) - 如果
\(a \mid b\),那么 b/a 是一个整数 - 如果 a 不整除 b,我们记作
\(a\nmid b\) \textbf{良序原理( Well Ordering
principle)}:\(每个非空的自然数集都有最小的元素\)

\subsubsection{1.1.2 同余关系}\label{ux540cux4f59ux5173ux7cfb}

\textbf{定义}:\(如果 a 和 b 是整数,且 m 是正整数,那么当 m 整除 a−b 时,称 a 与 b 在模 m 下同余,记作 a \equiv b (mod m).\)

\begin{itemize}
\tightlist
\item
  符号 \(a \equiv b (\mod m)\) 表示 a 与 b 在模 m 下同余.
\item
  我们称 \(a \equiv b (\mod m)\) 为一个同余关系,且 m 是其模数.
\item
  两个整数在模 m 意义下同余当且仅当它们除以 m 后的余数相同.
\item
  如果 a 不与 b 在模 m 意义下同余,我们记作 \(a \not\equiv b(\mod m)\).
\end{itemize}

同余关系有:自反性,对成型,传递性。

在 \(a\equiv b (\mod m)\) 和 \(a\mod m=b\) 中,mod 的用法是不同的

\begin{itemize}
\tightlist
\item
  \(a\equiv b (\mod m)\) 是整数集上的一个关系,表示 a 和 b 在模 m
  意义下同余(类似于 '' 等号 ``).
\item
  在 \(a\mod b\) 中,``mod'' 表示一个函数,该函数返回 a 除以 m
  后的余数(类似于 '' 除号 ``).
\end{itemize}

设 m 为正整数,a 和 b 为整数,

则

\[
\huge
(a + b) (\mod m) = ((a\mod m) + (b\mod m))\mod m 
\]

且

\[
\huge
ab\mod m = ((a\mod m) (b\mod m))\mod m
\]

\subsubsection{1.1.3
二进制模幂算法}\label{ux4e8cux8fdbux5236ux6a21ux5e42ux7b97ux6cd5}

使用 n 的二进制展开, \(n = (a_{k-1},…,a_1,a_0)_2\) , 来计算 \(b^n\) .

注意到:
\(\large b^n=b^{a_{k-1}\cdot 2^{k+1}+…+a_1\cdot 2+a_0}=(b^{2^{k-1}})^{a_k-1}…(b^2)^{a_1}b^{a_0}\)

该算法依次计算 \(b\mod m\), \(b_2\mod m\), \(b_4\mod m\), \ldots, 然后将
\(a_j = 1\) 的项相乘.

\subsubsection{1.1.4 群}\label{ux7fa4}

群, \(G\), 是一个元素集合,带有一个关联的二元运算,记作 * , 有时被表示为
\(\{G, * \}\),且

\emph{G} 中元素满足以下性质:

\begin{itemize}
\tightlist
\item
  \textbf{封闭性(Closure)}:若 \(a \in G\) 且 \(b \in G\),则
  \(a \cdot b \in G\)
\item
  \textbf{结合律(associative)}:对于\emph{G}中的任意元素 a, b 和𝑐,
  满足 \(a \cdot (b \cdot c) = (a \cdot b) \cdot c\)
\item
  \textbf{单位元(identity element)}:存在元素
  \(e \in G\),使得对于𝐺中的任意元素 a 满足
  \(a \cdot 𝑒 = 𝑒 \cdot a = a\)
\item
  \textbf{逆元(inverse element)}:对于 \emph{G} 中的任意元素 a,存在
  \emph{G} 中的另一元素 a 使得 \(a\cdot a'=a'\cdot a=e\)
\end{itemize}

一个群若满足交换律,则被称为\textbf{交换群(或阿贝尔群, abelian
group)}:

\begin{itemize}
\tightlist
\item
  \textbf{交换律(Commutative)}:\(对于𝐺中的任意元素a和b, 满足a : b = b : a\)
\end{itemize}

一个群 G 称为\textbf{循环群(cyclic group)},如果 \(G\)
中的每一个元素都可以表示为某个固定元素 \(g\) 的幂 \(g^k\) (\(k\)
是一个整数)。这个固定元素 \(g\) 被称为\textbf{生成元(generator)}。

\begin{itemize}
\tightlist
\item
  无论是有限循环群,还是无限循环群,\textbf{循环群总是交换群(abelian)}
\end{itemize}

\subsubsection{1.1.5 环}\label{ux73af}

环 \{\(\mathbb{R}\), +,
×\},是一个具有两个二元运算的元素集合,分别称为加法(+)和乘法(×),且满足以下条件:

\begin{itemize}
\tightlist
\item
  \{𝑅, +\}是一个交换群,其单位元写作 0,a 的逆元为−a
\item
  \textbf{乘法下封闭( Closure under multiplication )}:若
  \(a \in \mathbb{R}\) 且 \(b \in \mathbb{R}\),则
  \(a×b \in \mathbb{R}\)
\item
  \textbf{乘法结合律( associativity of multiplication
  )}:对于𝑅中任意元素 a, b 和𝑐, 满足 \(a× b×𝑐 = (a×b)×𝑐\)
\item
  \textbf{乘法分配律( Distributive laws)}:

  \begin{itemize}
  \tightlist
  \item
    对于𝑅中任意元素 a, b 和𝑐, 满足 \(a× b + 𝑐 = a×b + a×𝑐\)
  \item
    对于𝑅中任意元素 a, b 和𝑐, 满足 \((a + b)×𝑐 = a×𝑐 + b×𝑐\)
  \end{itemize}
\end{itemize}

\[
\begin{matrix}
&不要求存在乘法单位元 \\
&不要求存在乘法逆元 \\
&不要求满足乘法交换律
\end{matrix}
\]

若一个环满足乘法交换律,则称其为交换环:

\begin{itemize}
\tightlist
\item
  \textbf{乘法交换律( associativity of multiplication )}:对于
  𝑅中任意元素 a 和 b, 满足 ab = ba 若一个环满足以下条件,则称其为整环:
\item
  \textbf{乘法单位元(multiplicative identity)}:存在元素
  \(1 \in \mathbb{R}\) ,使得对于𝑅中的任意元素 a 满足 \(a×1 = 1×a = a\)
\item
  \textbf{无零因子( No zero divisors )}:对于𝑅中的任意元素 a 和 b,若
  \(ab = 0\),则必有 \(a = 0 或 b = 0\) 整环示例: 整数集合 ℤ
  在一般的加法和乘法下
\end{itemize}

\begin{quote}
(1)加法下封闭性: \(a + b \in \mathbb{R}\) (2)加法结合律:
\(a + b + 𝑐 = a + (b + 𝑐)\) (3)加法单位元:\(a + 0 = 0 + a = a\)
(4)加法逆元:\(−a\) (5)加法交换律:\(a + b = b + a\)
(6)乘法下封闭性:\(ab \in \mathbb{R}\)
(7)乘法结合律:\(a× bc = (ab)×c\)
(8)乘法分配律:\(a× b+𝑐c= ab+ac,(a+b)×c= ac+bc\)
(9)乘法交换律:\(ab = ba\) (10)乘法单位元:\(a×1 = 1×a = a\)
(11)无乘法零因子:\(ab = 0 ⟹ (a = 0) \lor  (b = 0)\)
\end{quote}

\subsubsection{1.1.6 域}\label{ux57df}

域 \{\emph{F}, +, ×\}是一个存在乘法逆元的整环,也即满足以下性质:

\begin{itemize}
\tightlist
\item
  \{\emph{F}, +, ×\}是整环,满足性质 (1)\textasciitilde(11)
\item
  性质 (12)- 存在\textbf{乘法逆元(multiplicative inverse)}:
  对于\emph{F}中任意\textbf{非 0} 元素 a,存在\emph{F}中的另一元素
  \(a^{-1}\) 使得 \(a×a^{-1} = a^{-1}×a=1\)
\item
  性质 (4)- 存在\textbf{加法逆元(additive inverse)}: 对于𝐹中任意元素
  \(a\), 存在 \emph{F} 中的另一元素 \(−a\) 使得 \(a+(−a) =(−a) +a=0\)
\end{itemize}

\textbf{除法定义}: \(a/b = a(b^{-1})\)
!{[}{[}群、环、域的包含示意图.png{]}{]}

\subsection{1.2
素数和最大公约数}\label{ux7d20ux6570ux548cux6700ux5927ux516cux7ea6ux6570}

\subsubsection{1.2.1 素数}\label{ux7d20ux6570}

\textbf{定义}:对于一个大于 1的正整数 p,若 p 的正因子只有 1 和 p
本身,则称 p 为\textbf{素数(prime)} 。

\subsubsection{1.2.2
算术基本定理}\label{ux7b97ux672fux57faux672cux5b9aux7406}

\textbf{定理}: 每一个大于 1
的正整数都可以唯一地表示为一个素数,或者表示为两个或更多素数的乘积,其中素因数按照大小不减的顺序排列.

\subsubsection{1.2.3 最大公约数}\label{ux6700ux5927ux516cux7ea6ux6570}

\textbf{定义}: 设 a 和 b 为整数,且不全为零。能够同时整除 a 和 b
的最大整数 𝑑 称为 a 和 b 的最大公约数,记作 \(\gcd(a, b)\).

\textbf{最大公约数表示成一个线性组合}

\begin{itemize}
\tightlist
\item
  \textbf{定理}: 若 a 和 b 是任意整数且不全为零,则 \(\gcd(a, b)\)
  是集合 \({ax + by ∶ x, y \in \mathbb{Z}}\) 中的最小正整数元素
\item
  \textbf{bézout's Theorem (贝祖定理)}:若 a 和 b 是正整数, 则存在整数
  𝑠 和 𝑡 使得 \(\gcd(a, b) = sa + tb\).
\item
  \textbf{定义}: 若 a 和 b 是正整数,则使得 \(\gcd(a, b) = sa + tb\)
  的整数 𝑠 和 𝑡 称为 a 和 b 的贝祖系数.
\end{itemize}

\paragraph{1.2.3.1
用素因子分解法找最大公约数}\label{ux7528ux7d20ux56e0ux5b50ux5206ux89e3ux6cd5ux627eux6700ux5927ux516cux7ea6ux6570}

假设 a 和 b 的素因数分解是:

\[
\huge
a=p_1^{a_1}p_2^{a_2}...p_n^{a_n}
\]

\[
\huge
b=p_1^{b_1}p_2^{b_2}...p_n^{b_n}
\]

其中每个指数都是非负整数,且两个素因数分解中出现的所有素数都包含在两者中.

\paragraph{1.2.3.2
欧几里得算法、辗转相除法}\label{ux6b27ux51e0ux91ccux5f97ux7b97ux6cd5ux8f97ux8f6cux76f8ux9664ux6cd5}

其基于的理念是,当 \(a \geq b\) 且 \(c = a \mod b\),则
\(\gcd(a, b) = \gcd(b, c)\).

\paragraph{1.2.3.3 求贝祖系数}\label{ux6c42ux8d1dux7956ux7cfbux6570}

\textbf{两步法} - 第 1 步:使用欧几里得算法求解最大公约数 - 第 2
步:通过回代将最大公约数表示为原始两个数的线性组合

\textbf{扩展欧几里得算法} 求满足 \(xa + yb = gcd(a, b)\) 的 x 和 y,假设
a ≥ b 由于

\[
\huge
gcd (a, b) = gcd(b, a\mod b)
\]

存在 x' 和 y' 满足

\[
\huge
x'b + y'( a\mod b) = gcd(b, a\mod b)
\]

\subsubsection{1.2.4 最小公倍数}\label{ux6700ux5c0fux516cux500dux6570}

\textbf{定义}:正整数 a 和 b 的最小公倍数是同时能被 a 和 b
整除的最小正整数,记作 \(lcm(a, b)\).
最小公倍数也可以通过素因数分解来计算.

\[
\huge
lcm(a,b)=p_1^{max(a_1,b_1)}p_2^{max(a_2,b_2)}...p_n^{max(a_n,b_n)}
\]

\textbf{定理}: 若 \(a, b\) 为正整数,则
\(ab = \gcd(a, b) \cdot \text{lcm}(a, b)\)

\subsubsection{1.2.5
和素数有关的各项定理}\label{ux548cux7d20ux6570ux6709ux5173ux7684ux5404ux9879ux5b9aux7406}

\textbf{素数测试及试除法} - \textbf{定理}:若 \(a\) 是合数, 则 \(a\)
必有小于或等于 \(\sqrt a\) 的因子. - \textbf{试除法}:依次将
\(\leq \sqrt a\) 的素数除 \(a\) 来判断 \(a\) 是否素数

\textbf{埃拉托斯特尼筛法}

\textbf{无穷素数} - \textbf{定理}: 素数的个数是无限的.

\textbf{梅森素数} - \textbf{定义}: 形如 \(2^p-1\) 的素数 , 且 p
是素数,被称为\textbf{梅森素数(mersenne primes)} - 截至 2024 年 10
月,已知 52 个梅森素数,其中最大的一个是 \(2^{136279841} − 1\) ,它有近
4100 多万个十进制数字

\textbf{素数定理(prime Number Theorem)} - 不超过 \(x\) 的素数个数与
\(x/ \ln x\) 的比率随着 \(x\) 的增大而趋近于 1 - 没有一个简单的函数 𝑓(𝑛)
使得 𝑓(𝑛) 对所有正整数 𝑛 都为素数

\subsubsection{1.2.6
关于素数的猜想}\label{ux5173ux4e8eux7d20ux6570ux7684ux731cux60f3}

\textbf{哥德巴赫猜想}:每个大于 2 的偶整数 𝑛
都是两个素数之和。该猜想已经通过计算机验证了所有不超过
\(1.6\cdot 10^{18}\) 的正偶整数。多数数学家相信这一猜想是正确的.
\textbf{孪生素数猜想}:孪生素数猜想认为存在无穷多对孪生素数。孪生素数是指相差
2 的素数对。例子包括 3 和 5,5 和 7,11 和 13 等。

\subsubsection{1.2.7 互素}\label{ux4e92ux7d20}

\textbf{定义}:如果两个整数 a 和 b 的最大公约数为 1,则称 a 和 b 互素.
\textbf{定义}:我们称 \(a_1, a_2, ⋯ , a_n\),两两互素当
\(gcd(a_i, a_j) = 1\) 对于任意 1 ≤ i \textless{} j ≤ 𝑛都成立

\subsubsection{1.2.8
同余式除以整数}\label{ux540cux4f59ux5f0fux9664ux4ee5ux6574ux6570}

\textbf{定理} : 设 m 为正整数,a、b 和 c 为整数. 若
\(ac \equiv bc (\mod m)\) 且 \(\gcd(c, m) = 1\), 则
\(a \equiv b (\mod m)\).

\subsection{1.3
求解同余方程}\label{ux6c42ux89e3ux540cux4f59ux65b9ux7a0b}

\subsubsection{1.3.1 线性同余式}\label{ux7ebfux6027ux540cux4f59ux5f0f}

\textbf{定义}:形如 \(ax \equiv b(\mod  m)\) 的同余式,其中 m
为正整数,a 和 b 是整数,x 是变量,称为线性同余式

\subsubsection{1.3.2 逆元}\label{ux9006ux5143}

\textbf{定义}:若整数 \(\bar a\) 满足 \(a\bar a \equiv 1\mod m\) ,则称
\(\bar a\) 为 a 模 m 的逆元. \textbf{定理 1}:若 a 和 m 是互素的整数且 m
\textgreater{} 1 ,则 a 在模 m 下有唯一逆元。

\paragraph{求解逆元通用解法}\label{ux6c42ux89e3ux9006ux5143ux901aux7528ux89e3ux6cd5}

\textbf{目标}:求解 a 在模 m 下的逆元: - 步骤 1:利用 (扩展)
欧几里得算法求解 \(xa + ym = gcd(a, m) = 1\) - 步骤 2:两边同时模
\(m\),可得 \(xa \equiv 1 (\mod m)\)

\subsubsection{1.3.3
用逆元求解线性同余式}\label{ux7528ux9006ux5143ux6c42ux89e3ux7ebfux6027ux540cux4f59ux5f0f}

通过两边同时乘以 \(\bar a\) 来解方程 \(ax \equiv b(\mod m)\)

\subsubsection{1.3.4
中国剩余定理}\label{ux4e2dux56fdux5269ux4f59ux5b9aux7406}

有若干物, 其数未知。除以 3,余数为 2;除以 5,余数为 3;除以 7,余数为
2。这些物的数量是多少?

这个谜题可以被转化为解以下同余方程组的问题:

\[
\huge
\begin{cases}
x \equiv 2 ( \mod 3)\\
x \equiv 3 ( \mod 5)\\
x \equiv 2 ( \mod 7)\\
\end{cases}
\]

!{[}{[}中国剩余定理求解.png{]}{]}

\textbf{定理 2: (The Chinese Remainder Theorem)} 设
\(m_1, m_2, … , m_𝑛\) 是两两互素且大于 1 的正整数,\(a_1, a_2, … , a_n\)
是任意整数. 以下同余方程组

\[
\huge
\begin{matrix}
x \equiv a_1 (\mod m_1)\\
x \equiv a_2 (\mod m_2)\\
\cdot\\
\cdot\\
\cdot\\
x \equiv a_𝑛 (\mod m_𝑛)\\
\end{matrix}
\]

\(有唯一的解模 m = m_1m_2 … m_𝑛.\)

\((即存在一个解 x,使得 0 ≤ x < m,且所有其他解与该解模 m 同余.)\)

\textbf{求解过程:}

\[
\huge
\begin{cases}
x \equiv a_1 ( \mod m_1)\\
x \equiv a_2 ( \mod m_2)\\
\cdot\\
\cdot\\
\cdot\\
x \equiv a_n ( \mod m_n)\\
\end{cases}
\]

\begin{itemize}
\tightlist
\item
  步骤 1. \(计算所有模数的积:m = m_1m_2 ⋯ m_n,\)
\item
  步骤 2. \(针对每个同余方程 x \equiv ak (\mod m_k),\)

  \begin{itemize}
  \tightlist
  \item
    \begin{enumerate}
    \def\labelenumi{\alph{enumi}.}
    \tightlist
    \item
      \(计算 m_k = m/m_k\)
    \end{enumerate}
  \item
    \begin{enumerate}
    \def\labelenumi{\alph{enumi}.}
    \setcounter{enumi}{1}
    \tightlist
    \item
      \(利用(扩展)欧几里得算法计算m_k在模m_k下的逆元,记为m_k^{-1}.\)
    \end{enumerate}
  \end{itemize}
\item
  步骤 3. \(\large x=(\sum\limits_{i=1}^{n} a_im_im_i^{-1} )\mod m\)
\end{itemize}

\subsubsection{1.3.5 回代法}\label{ux56deux4ee3ux6cd5}

\textbf{将每个同余方程转化为等式,代入变量的值到另一个同余方程中,并持续进行这个过程,直到解决所有同余方程}

\begin{quote}
\textbf{例}: 使用回代法找到所有满足以下条件的整数 𝑥使得
\(x \equiv 1 (\mod 5), x \equiv 2 (\mod 6), x \equiv 3 (\mod 7).\)
\end{quote}

\begin{quote}
\textbf{解}: 第一个同余方程可以重写为𝑥 = 5𝑡 + 1,其中 𝑡 是一个整数.
代入第二个同余方程得到 \(5𝑡 + 1 \equiv 2 (mod 6)\).

解这个方程得𝑡 ≡ 5 (mod 6),因此 𝑡 = 6𝑢 + 5,其中 𝑢 是一个整数. 代回𝑥 =
5𝑡 + 1, 解得 𝑥 = 5(6𝑢 + 5) + 1 = 30𝑢 + 26. 再代入第三个方程得到 30𝑢 + 26
≡ 3 (mod 7). 解得𝑢 ≡ 6 (mod 7). 因此𝑢 = 7𝑣 + 6,其中 𝑣 是一个整数. 将 𝑢
= 7𝑣 + 6 代入 𝑥 = 30𝑢 + 26 中得到𝑥 = 30(7𝑣 + 6) + 26 = 210𝑢 + 206.
将这个结果转化为同余方程,我们得到解𝑥 ≡ 206 (mod 210)
\end{quote}

\subsubsection{1.3.6 Wilson 定理}\label{wilson-ux5b9aux7406}

\(p是素数当且仅当 (p − 1)! ≡ −1(\mod p)\)

\subsubsection{1.3.7
欧拉函数与欧拉定理}\label{ux6b27ux62c9ux51fdux6570ux4e0eux6b27ux62c9ux5b9aux7406}

\textbf{欧拉函数𝜑(𝑛)}:比𝑛小且与𝑛互质的正整数的个数. 例:𝜑(6) = 2 {[}1,
5{]}; 𝜑(7) = 6 {[}1, 2, 3, 4, 5, 6{]}
\textbf{欧拉定理}:\(若 gcd(a, 𝑛) = 1,则a^{𝜑(n)} ≡ 1 (\mod 𝑛).\)
\textbf{欧拉定理推论}:\(若 gcd(a, 𝑛) = 1,则a^{x\mod 𝜑(n)} ≡ a (mod 𝑛).\)

\subsubsection{1.3.8 费马小定理}\label{ux8d39ux9a6cux5c0fux5b9aux7406}

\textbf{定理
3}:\(如果 p 是一个素数且 a 是一个不被 p 整除的整数, 则有 a^{p-1} ≡ 1\ (\mod p)\)

\subsubsection{1.3.9 伪素数}\label{ux4f2aux7d20ux6570}

给定一个正整数 𝑛,如果 \(2n-1 ≡ 1 (\mod n)\):

\begin{itemize}
\tightlist
\item
  若𝑛不满足该同余方程,则𝑛是合数.
\item
  若𝑛满足该同余方程,则𝑛可能是素数,也可能是基数 2 的伪素数
\end{itemize}

\subsubsection{1.3.10 原根}\label{ux539fux6839}

\textbf{定义}:一个素数 p 的原根是指:
\(ℤ_p中的一个整数 𝑟,使得 ℤ_p中的每一个非零元素都是 𝑟 的某个幂\)
\textbf{重要事实}:对于每个素数 p,总存在一个原根

\subsubsection{1.3.11 离散对数}\label{ux79bbux6563ux5bf9ux6570}

假设 p 是一个素数,𝑟 是模 p 的一个原根。若 a 是一个位于 1 和 p−1
之间的整数,

即 a 是 \(ℤ_p\) 的一个元素,则存在一个唯一的指数 𝑒 使得 \(𝑟^e = a\) 在
\(ℤ_p\) 中成立,也即 \(𝑟^e\mod p = a\).

定义:设 p 是一个素数,𝑟 是模 p 的一个原根,且 a 是位于 1 到 p−1
之间的整数。若 \(𝑟^e\mod p = a\),且 1 ≤ 𝑒 ≤ p − 1,我们称 𝑒 为模 p 以 𝑟
为底数的 a 的离散对数, 写作 \(log_r a = 𝑒\) (其中素数 p 是默认的).

\subsection{1.4
同余式的应用}\label{ux540cux4f59ux5f0fux7684ux5e94ux7528}

\subsubsection{1.4.1 哈希函数}\label{ux54c8ux5e0cux51fdux6570}

\textbf{定义}: 哈希函数 ℎ 将具有键 k 的记录分配到存储位置 ℎ(k).
一个常见的哈希函数是 \(ℎ(k) = k\mod m\),其中 m 是存储位置的总数.

\subsubsection{1.4.2 伪随机数}\label{ux4f2aux968fux673aux6570}

线性同余法是生成伪随机数的一种常用方法.

\begin{itemize}
\tightlist
\item
  生成伪随机数需要四个整数:模数 m、乘数 a、增量 𝑐、以及种子 \(x_0\),
  其中 \(2\ ≤\ a\ <\ m,\ 0\ ≤\ 𝑐\ <\ m,\ 0\ ≤\ x_0\ <\ m.\)
\item
  我们通过以下递归定义的函数生成伪随机数序列 \(\{x_𝑛\}\):
  \(x_{n+1} = (ax_n + 𝑐) mod m\) 若需介于 0 和 1
  之间的伪随机数,可将生成的数字除以模数, 即 \(x_𝑛 /m\)
\end{itemize}

\subsection{1.5 密码学}\label{ux5bc6ux7801ux5b66}

\subsubsection{其他的太简单这里只写 RSa
密码系统}\label{ux5176ux4ed6ux7684ux592aux7b80ux5355ux8fd9ux91ccux53eaux5199-rsa-ux5bc6ux7801ux7cfbux7edf}

\subsubsection{1.5.1 RSa 密码系统}\label{rsa-ux5bc6ux7801ux7cfbux7edf}

设 p 和𝑞是两个不相等的大素数,𝑛 = p𝑞。

选择一个与 𝜑(𝑛) 互素的正整数 𝑤,并设 𝑑 为 𝑤 在模𝜑(𝑛) 下的逆

\begin{itemize}
\tightlist
\item
  加密秘钥:𝑤 和 𝑛
\item
  解密秘钥:𝑑 和 𝑛
\item
  需保密数据:p,𝑞,𝜑(𝑛) ,𝑑 设 m 是待加密数据块,且 m \textless{} 𝑛:
\item
  加密算法:\(𝑐 = 𝐸 (m) = m^w\mod 𝑛\)
\item
  解密算法:\(𝐷 (𝑐) = 𝑐^d\mod 𝑛\)
\end{itemize}

\subsubsection{1.5.2
加密协议:密钥交换}\label{ux52a0ux5bc6ux534fux8baeux5bc6ux94a5ux4ea4ux6362}

密钥交换是一种加密协议:双方可在没有任何过去共享秘密信息的情况下,
通过不安全的通道交换一个秘密密钥。

下面通过示例描述了 Diffie-Hellman 密钥协议.

\begin{enumerate}
\def\labelenumi{\arabic{enumi}.}
\tightlist
\item
  假设 alice 和 bob 希望共享一个共同的密钥.
\item
  alice 和 bob 同意使用一个素数 p 和 p 的一个原根 a.
\item
  alice 选择一个秘密整数 k1,并将 \(a^{k1}\mod p\) 发送给 bob.
\item
  bob 选择一个秘密整数 k2,并将 \(a^{k2}\mod p\) 发送给 alice.
\item
  alice 计算 \((a^{k2} )^{k1}\) mod p.
\item
  bob 计算 \((a^{k1})^{k2}\) mod p. 在协议结束时, alice 和 bob
  都有他们的共享密钥 (ak2)k1 mod p = (ak1)k2 mod p
\end{enumerate}

\subsubsection{1.5.3 数字签名}\label{ux6570ux5b57ux7b7eux540d}

bob 为确保收到的消息来自 alice

\begin{itemize}
\tightlist
\item
  alice 使用解密函数加密,发送 \(c=D_{(n,d)}(m)\)
\item
  bob 使用加密函数解密,得到 \(m'=E_{(n,w)}(c)\) 若
  \(m'=m\),则可以确信消息来自 \(alice\)
\end{itemize}

\subsubsection{1.5.4 同态加密}\label{ux540cux6001ux52a0ux5bc6}

\textbf{定义} 全同态加密指对密文经行操作等价于对明文经行操作,即满足性质

\[
\huge
\huge E((m_1+m_2)\times m_3)=(E(m_1)+E(m_2))\times E(m_3)
\]

\textbf{部分同态加密}即只对某些运算同态 - \textbf{加法同态}:

\[
\huge E(m_1+m_2)=E(m_1)+E(m_2)
\]

\begin{itemize}
\tightlist
\item
  \textbf{乘法同态}:
\end{itemize}

\[
\huge E(m_1\times m_2)=E(m_1)\times E(m_2)
\]

\begin{quote}
RSa 密码系统是乘法同态,非加法同态
\end{quote}

\subsubsection{1.5.5 Schnorr Zkp
零知识证明协议}\label{schnorr-zkp-ux96f6ux77e5ux8bc6ux8bc1ux660eux534fux8bae}

证明者向验证者证明其知晓离散对数,但不透露具体值:

选取一个素数 \(p\) 及其原根 \(g\),证明者有私钥 \(x<p\),公钥
\(y=g^x\mod{p}\),重复 \(n\) 轮以下操作:

\begin{itemize}
\tightlist
\item
  证明者随机选取一个小于 \(p\) 的整数 \(r\),发送 \(a=g^r\mod p\)
\item
  验证者发送一个随机值 \((e=0或1)\)
\item
  证明者发送 \(s=(r+ex)\mod{(p-1)}\)
\item
  验证者验证 \(g^s=ay^e\mod p\) 是否成立
\end{itemize}

\begin{quote}
重复 \(n\) 轮下,若证明者不知道离散对数,则验证成功率为
\(\left(\frac{1}{2}\right)^n\)
\end{quote}

\subsection{1.6 余数系统}\label{ux4f59ux6570ux7cfbux7edf}

\subsubsection{定义}\label{ux5b9aux4e49}

利用中国剩余定理,可以将一个大整数表示为元组

\textbf{定义} 大整数 \(x\) 在余数系统下可表示为

\[
\huge x=(x_1\mid x_2\mid \cdots\mid x_k)\text{RNS}(p_1\mid p_2\mid \cdots\mid p_k)
\]

其中

\[
\huge x_i=x\mod p_i
\]

且对任意 \(i,j\),有 \(p_i,p_j\) 互素

\subsubsection{运算}\label{ux8fd0ux7b97}

\[
\large x+y=((x_1+y_1)\mod p_1\mid (x_2+y_2)\mod p_2\mid \cdots(x_k+y_k)\mod p_k)
\]

\[
\large x\times y=((x_1\times y_1)\mod p_1\mid (x_2\times y_2)\mod p_2\mid \cdots(x_k\times y_k)\mod p_k)
\]

\section{第二章
组合计数基础}\label{ux7b2cux4e8cux7ae0-ux7ec4ux5408ux8ba1ux6570ux57faux7840}

\subsection{2.1 鸽巢原理}\label{ux9e3dux5de2ux539fux7406}

\textbf{定理} 将 \(k+1\) 个物体放入 \(k\)
个盒子中,则至少有一个盒子包含两个或更多物体

\textbf{广义鸽巢原理} 将 \(N\) 个物体放入 \(k\)
个盒子中,则至少有一个盒子包含至少 \(\lceil N/k\rceil\) 个物体

\subsection{2.2 排列组合}\label{ux6392ux5217ux7ec4ux5408}

\textbf{定义} 对于 \(n\) 个不同元素的集合,其 \(r\)- 排列的数量为

\[
\huge p(n,r)=\dfrac{n!}{(n-r)!}
\]

\textbf{定义} 对于 \(n\) 个不同元素的集合,其 \(r\)- 组合的数量为

\[
\huge C(n,r)或\binom{n}{r}=\dfrac{n!}{r!(n-r)!}
\]

\subsubsection{组合证明}\label{ux7ec4ux5408ux8bc1ux660e}

组合证明指使用以下方法之一来证明恒等式:

\begin{itemize}
\tightlist
\item
  \textbf{双计数证明}:使用计数论证来证明恒等式两边以不同的方式计算相同的对象
\item
  \textbf{双射证明}:展示恒等式两边所计数的对象集合之间存在双射关系
\end{itemize}

\paragraph{二项式定理}\label{ux4e8cux9879ux5f0fux5b9aux7406}

\[
\huge (x+y)^n=\sum_{k=0}^{n}\binom{n}{k}x^{n-k}y^k
\]

\paragraph{帕斯卡恒等式}\label{ux5e15ux65afux5361ux6052ux7b49ux5f0f}

\[
\huge \binom{n+1}{k}=\binom{n}{k}+\binom{n}{k-1}
\]

\subsection{2.3
广义排列组合}\label{ux5e7fux4e49ux6392ux5217ux7ec4ux5408}

\subsubsection{2.3.1
带有重复对象的排列组合}\label{ux5e26ux6709ux91cdux590dux5bf9ux8c61ux7684ux6392ux5217ux7ec4ux5408}

\begin{longtable}[]{@{}lll@{}}
\toprule\noalign{}
类型 & 是否允许重复 & 公式 \\
\midrule\noalign{}
\endhead
\bottomrule\noalign{}
\endlastfoot
\(r\)- 排列 & 否 & \(\huge p(n,r)\) \\
\(r\)- 组合 & 否 & \(\huge\binom{n}{r}\) \\
\(r\)- 排列 & 是 & \(\huge n^r\) \\
\(r\)- 组合 & 是 & \(\huge \binom{n+r-1}{r}\) \\
\end{longtable}

\subsubsection{2.3.2
带有不可区分对象的排列}\label{ux5e26ux6709ux4e0dux53efux533aux5206ux5bf9ux8c61ux7684ux6392ux5217}

\textbf{定理} \(n\) 个对象分为 \(k\) 类,每类分别有 \(n_1,n_2,…,n_k\)
个对象,不同的排列数量是

\[
\huge \dfrac{n!}{n_1!n_2!\cdots n_k!}
\]

\begin{quote}
这里的不可区分指的是同一类中的不可区分
\end{quote}

\subsubsection{2.3.3
将对象分配到盒子中}\label{ux5c06ux5bf9ux8c61ux5206ux914dux5230ux76d2ux5b50ux4e2d}

\begin{longtable}[]{@{}
  >{\raggedright\arraybackslash}p{(\linewidth - 4\tabcolsep) * \real{0.1803}}
  >{\raggedright\arraybackslash}p{(\linewidth - 4\tabcolsep) * \real{0.1803}}
  >{\raggedright\arraybackslash}p{(\linewidth - 4\tabcolsep) * \real{0.6393}}@{}}
\toprule\noalign{}
\begin{minipage}[b]{\linewidth}\raggedright
\(n\) 物体是否可区分
\end{minipage} & \begin{minipage}[b]{\linewidth}\raggedright
\(k\) 盒子是否可区分
\end{minipage} & \begin{minipage}[b]{\linewidth}\raggedright
公式
\end{minipage} \\
\midrule\noalign{}
\endhead
\bottomrule\noalign{}
\endlastfoot
是 & 是 & \(\huge \dfrac{n!}{n_1!n_2!\cdots n_k!}\) \\
否 & 是 & \(\huge \binom{n+k-1}{k}\) \\
是 & 否 & 无简单封闭公式 \\
否 & 否 & 无简单封闭公式 \\
\end{longtable}

\begin{quote}
这里的区分指的是所有物体或盒子之间的区分
\end{quote}

\subsection{2.4 母函数}\label{ux6bcdux51fdux6570}

\textbf{定义} 对于序列 \(a_0,a_1,a_2,…\),函数

\[
\huge G(x)=a_0+a_1x+a_2x^2+\cdots
\]

称为序列的母函数

\begin{quote}
母函数利用多项式乘法的过程来模拟组合
\end{quote}

母函数是这样一种方法,对于对象
\(k\),其母函数的指数为所有可能的选取的个数,各项系数为该选取下对应的方法数。将所有对象的母函数乘起来,得到的描述所有对象做组合的状态空间。

\textbf{例} 有 8 个白球和 5
个黑球,要求选取偶数个白球和不少于两个黑球,有多少组合方式 \textbf{解}
白球的母函数为

\[
\huge G_w(x)=1+x^2+x^4+x^6+x^8
\]

黑球的母函数为

\[
\huge G_b(x)=x^2+x^3+x^4+x^5+x^6
\]

取乘积

\[
\huge G=G_w(x)G_b(x)
\]

其中 \(x^k\) 的系数即为 \(k\) 个球的组合的方法数

\textbf{例} 若上例中白球不可区分,则有多少组合方式 \textbf{解}
白球母函数改写为

\[
\huge G_w(x)=\binom{8}{0}+\binom{8}{2}x^2+\cdots+\binom{8}{8}x^8
\]

\subsubsection{2.4.1 整数拆分}\label{ux6574ux6570ux62c6ux5206}

多项式相乘时指数相加,利用指数来表征整数的组合

\paragraph{2.4.1.1 无序拆分}\label{ux65e0ux5e8fux62c6ux5206}

\textbf{例} 整数 \(a_1,a_2,…,a_n\) 分别有 \(k_1,k_2,…,k_n\) 个,求整数
\(N\) 有几种拆分方式 \textbf{解} 整数 \(a_i\) 对应的母函数为

\[
\huge G_i(x)=1+x^{a_i}+x^{2a_i}+\cdots+x^{k_ia_i}=\dfrac{1-x^{(k_i+1)a_i}}{1-x^{a_i}}
\]

若 \(a_i\) 有无穷个,则

\[
\huge G_i(x)=\dfrac{1}{1-x^{a_i}}
\]

将所有母函数相乘

\[
\huge G(x)=\prod_{i=1}^n{G_i(x)}
\]

其中 \(x^N\) 的系数即为 \(N\) 的拆分方式数

\paragraph{2.4.1.2 有序拆分}\label{ux6709ux5e8fux62c6ux5206}

\textbf{例(定项拆分)} 将 \(N\) 有序拆分为 \(r\) 个非 \(0\) 部分
\textbf{解} \textbf{方法 1(隔板法)} 这相当于在 \(N\) 个球中插入
\(r-1\) 个隔板,结果为 \(\binom{N-1}{r-1}\) \textbf{方法 2(母函数)}
每个部分的数对应的母函数为

\[
\huge G_i(x)=x+x^2+\cdots=\dfrac{x}{1-x}
\]

相乘得

\[
\huge G(x)=x^r(1-x)^{-r}=x^r\sum_{k=0}^{\infty}\binom{r+k-1}{r-1}x^k
\]

令 \(k+r=N\),有

\[
\huge G(x)=\sum_{N=r}^{\infty}\binom{N-1}{r-1}x^N
\]

\textbf{例(不定项拆分)} 将 \(N\) 有序拆分的方法数 \textbf{解} 只需将
\(N\) 的 \(r\)- 拆分累加即可

\[
\huge \sum_{r=1}^N\binom{N-1}{r-1}=\sum_{j=0}^{N-1}\binom{N-1}{j}=2^{N-1}
\]

\textbf{例} 将 \(N\) 有序拆分为无限个整数 \(a_1,a_2,…,a_n\) 的和
\textbf{解} 母函数为

\[
\huge G=1+(x^{a_1}+x^{a_2}+\cdots+x^{a_n})+(x^{a_1}+x^{a_2}+\cdots+x^{a_n})^2+\cdots
\]

\subsubsection{2.4.2
指数型母函数}\label{ux6307ux6570ux578bux6bcdux51fdux6570}

\textbf{定义} 对于序列 \(a_0,a_1,a_2,…\),函数

\[
\huge G_e(x)=a_0+\dfrac{a_1}{1!}x+\dfrac{a_2}{2!}x^2+\cdots
\]

称为序列的指数型母函数

\begin{quote}
用于解决从分为 \(k\) 类的对象中做 \(r\)-排列的问题
\end{quote}

\textbf{例} 现有 \(n\) 个元素 \(a_1,a_2,…,a_n\),分别重复
\(k_1,k_2,…,k_n\) 次,对其做 \(r\)- 排列 \textbf{解} \(a_i\)
对应的指数型母函数为

\[
\huge G_{ei}(x)=1+\dfrac{x}{1!}+\dfrac{x^2}{2!}+\cdots+\dfrac{x^{k_i}}{k_i!}
\]

若 \(a_i\) 有无穷个,则

\[
\huge G_{ei}(x)=e^x
\]

其 \(r\)- 排列的母函数为

\[
\huge G_e(x)=\prod_{i=1}^nG_{ei}(x)
\]

其中,\(\dfrac{x^r}{r!}\) 对应的系数为 \(r\)- 排列的方法数

\textbf{例} 由 \(1,2,3,4,5\) 组成 \(n\) 位数,其中 \(2,4\) 出现偶数次
\textbf{解} 指数型母函数为

\[
\huge
\begin{align}
G_e(x)&=\left(1+\dfrac{x^2}{2!}+\dfrac{x^4}{4!}+\cdots\right)^2\left(1+\dfrac{x}{1!}+\dfrac{x^2}{2!}+\cdots\right)^3\\
&=\left(\dfrac{e^x+e^{-x}}{2}\right)^2e^{3x}\\
&=\dfrac{1}{4}(e^{5x}+2e^{3x}+e^x)\\
&=\sum_{n=0}^\infty\dfrac{1}{4}(5^n+2\cdot3^n+1)\dfrac{x^n}{n!}
\end{align}
\]

即 \(n\) 位数有 \(a_n=\dfrac{1}{4}(5^n+2\cdot3^n+1)\) 种可能

\section{第三章
高级计数技术}\label{ux7b2cux4e09ux7ae0-ux9ad8ux7ea7ux8ba1ux6570ux6280ux672f}

\subsection{3.1
递归关系的应用}\label{ux9012ux5f52ux5173ux7cfbux7684ux5e94ux7528}

\subsubsection{3.1.1 递归关系}\label{ux9012ux5f52ux5173ux7cfb}

\begin{quote}
\textbf{定义}:数列 \(\{ a_n\}\) 的递归关系是一个方程,它将 \(a_n\)
表示为该数列之前的一个或多个项的函数,即
\(a_1, a_1, …,a_{n-1}\),其中𝑛为所有满足 \(n \geq n_0\) 的整数,\(n_0\)
是一个非负整数. 1.
如果一个数列的各项满足递归关系,则称该数列为递归关系的解. 2.
数列的初始条件指定了递归关系生效前的各项.
\end{quote}

\paragraph{3.1.1.1
斐波那契数列}\label{ux6590ux6ce2ux90a3ux5951ux6570ux5217}

\begin{quote}
\textbf{问题}
一对年轻的兔子(一公一母)被放置在一个岛上。兔子在满两个月前不会繁殖。两个月大后,每对兔子每个月都会产下一对兔子。假设兔子永远不会死亡,求在经过𝑛
个月后岛上兔子的对数的递归关系.
\end{quote}

\begin{quote}
\textbf{解答} 数列 \(\{f_n\}\) 满足递归关系
\(f_n = f_{n-1}+f_{n-2}\),适用于𝑛 ≥ 3 ,初始条件为 \(f_1 = 1\) 和
\(f_2 = 1\)。𝑛 个月后岛上的兔子对数由第𝑛 个斐波那契数给出.
\end{quote}

\paragraph{3.1.1.2 汉诺塔问题}\label{ux6c49ux8bfaux5854ux95eeux9898}

\begin{quote}
\textbf{规则}:你可以一次移动一个圆盘,从一个柱子移动到另一个柱子,只要大的圆盘永远不会放在小的圆盘上面.
\textbf{目标}:通过允许的移动,将所有圆盘最终放在第二个柱子上,按大小顺序排列,最大的在底部.
\end{quote}

\begin{quote}
\textbf{解答}:初始时,n 个圆盘在柱子 1 上。我们可以将顶部的 n−1
个圆盘按照谜题规则移动到柱子 3,这需要 \(H_{n−1}\) 次移动。我们使用 1
次移动将最大的圆盘转移到第二个柱子。然后,我们将 n−1 个圆盘从柱子 3
移到柱子 2,需要额外的 \(H_{n−1}\)
次移动。这个过程不能用更少的步骤完成。因此,
\end{quote}

\[
\huge H_n=2H_{n-1}+1
\]

\subsubsection{3.1.2
各种递归关系的例子}\label{ux5404ux79cdux9012ux5f52ux5173ux7cfbux7684ux4f8bux5b50}

\begin{itemize}
\tightlist
\item
  \(pn = (1.11)p_{n-1}\) 一阶线性齐次递归关系的形式
\item
  \(f_n = f_{n-1} + f_{n-2}\) 二阶线性齐次递归关系的形式
\item
  \(a_n=a_{n-1}+a_{n-2}^2\) 非线性
\item
  \(H_n = 2H_{n−1} + 1\) 非齐次
\item
  \(b_n = nb_{n−1}\) 系数不是常数
\end{itemize}

\subsection{3.2
线性递归关系的求解}\label{ux7ebfux6027ux9012ux5f52ux5173ux7cfbux7684ux6c42ux89e3}

\subsubsection{3.2.1
齐次递归关系}\label{ux9f50ux6b21ux9012ux5f52ux5173ux7cfb}

\begin{quote}
\textbf{定义}:一个具有常系数的 k 阶线性齐次递归关系的形式为
\(\large a_n= c_1a_{n−1} + c_2a_{n−2} + … + c_k a_{n−k}\) ,其中
\(\large c_1, c_2, ….,c_k\) 为实数,且 \(\large c_k ≠ 0\)
\end{quote}

\paragraph{3.2.1.1 二阶无重根}\label{ux4e8cux9636ux65e0ux91cdux6839}

设 \(c_1\) 和 \(c_2\) 为实数。假设方程 \(\large r^2 – c_1r – c_2 = 0\)
有两个不同的根 \(r_1\) 和 \(r_2\) 。则数列\{\(a_n\)\}是递归关系的解

\[
\huge
a_n=c_1a_{n-1}+c_2a_{n-2}
\]

当且仅当

\[
\huge
a_n=\alpha_1r_1^n+\alpha_1r_2^n
\]

对于 \(n = 0,1,2,…\) 成立,其中 \(\alpha_1\) 和 \(\alpha_2\) 是常数。

\paragraph{3.2.1.2 二阶有重根}\label{ux4e8cux9636ux6709ux91cdux6839}

设 \(c_1\) 和 \(c_2\) 为实数且 \(c_2 ≠ 0\) 。假设方程
\(\large r^2 –c_1r – c_2 = 0\) 有一个重复根 \(r_0\)
。那么数列\{\(a_n\)\}是递归关系 \(\large a_n = c_1a_{n−1} + c_2a_{n−2}\)
的解,当且仅当

\[
\huge 
a_n=\alpha_1r_0^n+\alpha_2nr_0^n
\]

对于 \(n = 0,1,2,…\) 成立,其中 \(\alpha_1\) 和 \(\alpha_2\) 是常数。

\paragraph{3.2.1.3 任意阶}\label{ux4efbux610fux9636}

\textbf{定理} 对于 \(k\) 阶常系数齐次递归关系

\[
\huge c_0a_n+c_1a_{n+1}+c_{2}a_{n+2}+\cdots+c_{k}a_{n+k}=0
\]

定义其特征方程为

\[
\huge c_kr^k+\cdots+c_{1}r+c_0=0
\]

若方程有 \(t\) 个根 \(r_1,…,r_t\) 且重数为
\(\alpha_1,…,\alpha_t\),则数列通项为

\[
\huge 
\begin{align}
a_n=&(C_{11}+C_{12}n+\cdots+C_{1\alpha_1}n^{\alpha_1-1})r_1^n+\\
&(C_{21}+C_{22}n+\cdots+C_{2\alpha_2}n^{\alpha_2-1})r_2^n+\\
&\cdots\\
&(C_{t1}+C_{t2}n+\cdots+C_{t\alpha_t}n^{\alpha_t-1})r_t^n\\
=&\sum_{i=1}^{t}\left(\sum_{j=1}^{\alpha_i}C_{ij}n^{j-1}\right)r_i^n
\end{align}
\]

其中 \(C_{ij}\) 为待定系数

\subsubsection{3.2.2
非齐次递归关系}\label{ux975eux9f50ux6b21ux9012ux5f52ux5173ux7cfb}

\paragraph{3.2.2.1
带有常系数的线性非齐次递归关系}\label{ux5e26ux6709ux5e38ux7cfbux6570ux7684ux7ebfux6027ux975eux9f50ux6b21ux9012ux5f52ux5173ux7cfb}

\textbf{定义}:带有常系数的线性非齐次递归关系的形式为:

\[
\huge
a_n = c_1a_{n−1} + c_2a_{n−2} + ….. + c_k a_{n−k} + F(n) 
\]

其中 \(c_1, c_2, ….,c_k\) 是实数, \(F(n)\) 是不全为零的函数,仅依赖𝑛.

递归关系

\[
\huge
a_n = c_1a_{n−1} + c_2a_{n−2} + ….. + c_k a_{n−k} 
\]

称为关联的齐次递归关系.

\textbf{定理}: 如果 \(\large \{a_n^{(p)}\}\)
是具有常系数的非齐次线性递归关系的一个特解

\[
\huge 
a_n=c_1a_{n-1}+c_2a_{n-1}+...+c_ka_{n-k}+F(n)
\]

那么每个解的形式为 \(\large \{a_n^{(p)} + a_n^{(h)}\}\),其中
\(\large \{a_n^{(h)}\}\) 是关联的齐次递归关系的解

\[
\huge
a_n = c_1a_{n-1}+c_2a_{n-1}+...+c_ka_{n-k}
\]

\subsection{3.3
分治算法与递归关系}\label{ux5206ux6cbbux7b97ux6cd5ux4e0eux9012ux5f52ux5173ux7cfb}

\subsubsection{3.3.1
分治算法与递归关系}\label{ux5206ux6cbbux7b97ux6cd5ux4e0eux9012ux5f52ux5173ux7cfb-1}

\textbf{定义}:分治算法通过首先将一个问题划分为一个或多个相同问题的小规模实例,然后利用这些较小问题的解来解决原始问题.
假设一个递归算法将一个规模为𝑛的问题划分为 a 个子问题.每个子问题的规模为
\(n/b\). 假设在合并步骤中需要𝑔(𝑛) 次额外操作.
则解决规模为𝑛的问题所需的操作数𝑓(𝑛) 满足以下递归关系:

\[
\huge 
f (n) = af (n / b) + g (n)
\]

这被称为分治递归关系.

\subsubsection{3.3.2 例子}\label{ux4f8bux5b50}

\paragraph{3.3.2.1 二分查找}\label{ux4e8cux5206ux67e5ux627e}

二分查找将对大小为𝑛的序列中的元素的搜索问题减少到对大小为 \(\large n/2\)
的序列中的搜索。实现这种减少需要两个比较;

\begin{itemize}
\tightlist
\item
  一个比较用于确定序列是否包含元素
\item
  另一个比较用于决定是搜索序列的上半部分还是下半部分 因此,如果
  \(\large 𝑓(𝑛)\) 是在大小为 \(\large n\)
  的序列中查找一个元素所需的比较次数,那么满足的递归关系是
\end{itemize}

\[
\huge
f (n) = f (n / 2) + 2
\]

当𝑛是偶数.

\paragraph{3.3.2.2 归并排序}\label{ux5f52ux5e76ux6392ux5e8f}

归并排序算法将一个规模为 \(\large n\)(假设 \(\large n\)
是偶数)的列表分成两个规模为 \(\large n/2\) 的子列表。它使用少于
\(\large n\) 次比较来合并这两个已排序的列表.

因此,排序一个规模为 \(\large n\) 的序列所需的比较次数 \(\large m(n)\)
满足以下递归关系

\[
\huge 
m (n) = 2m (n / 2) + n
\]

\paragraph{3.3.2.3
整数快速乘法}\label{ux6574ux6570ux5febux901fux4e58ux6cd5}

两个 \(\large 2n\) 位整数的乘法可以通过三个 \(\large n\)
位整数的乘法以及加法、减法和移位来完成。因此,如果 \(\large f(n)\)
是乘两个 \(\large n\) 位整数所需的总操作数,则

\[
\huge
f (2n) = 3 f (n) +Cn
\]

其中 \(\large Cn\) 表示总的位操作数;这些加法、减法和移位是 \(\large n\)
位操作的常数倍

\subsubsection{3.3.3
估算分治算法函数的大小}\label{ux4f30ux7b97ux5206ux6cbbux7b97ux6cd5ux51fdux6570ux7684ux5927ux5c0f}

\paragraph{定理 1}\label{ux5b9aux7406-1}

设𝑓(𝑛) 是一个递增函数,满足递归关系

\[
\huge
f(n) = a \cdot f(n/b) + c
\]

其中 \(\large n\) 是 \(\large b\)
的倍数,\(\large a \geq 1\),\(\large b\) 是大于 1 的整数,\(\large c\)
是一个正实数.

那么

\[
\huge
f(x) = \left\{ 
\begin{array}{cl} 
O(n^{\log_b{a}}) & \text{若 } a>1 \\
O(\log_b{n}) & \text{若 } a=1 
\end{array} 
\right.
\]

更进一步,

\[
\huge
f(x) = \left\{ 
\begin{array}{cl} 
\left(f(1)+\frac{c}{a-1}\right)n^{\log_b{a}}-\frac{c}{a-1} & \text{若 } a>1 \\
f(1)+c\log_b{n} & \text{若 } a=1 
\end{array} 
\right.
\]

\paragraph{定理 2}\label{ux5b9aux7406-2}

主定理(master Theorem):设𝑓(n) 是一个递增函数,满足递归关系

\[
\huge
f (n) = af (n / b) + cn^d
\]

当 \(\large n = bk\), \(\large k\) 为大于 1 的正整数, \(\large c\)
为正实数,\(\large d\) 为非负实数。

可具体化为

\[
\huge
f(x) = \left\{ 
\begin{array}{cl} 
O\left(n^d\right) & \text{若 } a<b^d \\
O\left(n^d\log{n}\right) & \text{若 } a=b^d \\
O\left(n^{\log_b{a}}\right) & \text{若 } a>b^d
\end{array} 
\right.
\]

\subsection{3.4 容斥原理}\label{ux5bb9ux65a5ux539fux7406}

\textbf{定理} 设 \(a_1,a_2,…,a_n\) 为有限集,则

\[
\huge
\begin{align}
|a_1\cup a_2\cup\cdots\cup a_n|=&\sum_{1\leq i\leq n}|a_i|-\sum_{1\leq i\leq j\leq n}|a_i\cap a_j|+\\
&\sum_{1\leq i\leq j\leq k\leq n}|a_i\cap a_j\cap a_k|-\cdots+\\
&(-1)^{n-1}|a_1\cap a_2\cap\cdots\cap a_n|
\end{align}
\]

\section{第四章
命题逻辑}\label{ux7b2cux56dbux7ae0-ux547dux9898ux903bux8f91}

\subsection{4.1 命题}\label{ux547dux9898}

一般来说,命题可分两种类型:

\begin{itemize}
\tightlist
\item
  原子命题 (简单命题):不能再分解为更为简单命题的命题。
\item
  复合命题:可以分解为更为简单命题的命题。而且这些简单命题之间是通过如``或者''、``并且''、``不''、``如果\ldots 则\ldots''、``当且仅当''等这样的关联词和标点符号复合而构成一个复合命题。
\end{itemize}

\subsection{4.2 各种联结词}\label{ux5404ux79cdux8054ux7ed3ux8bcd}

\subsubsection{4.2.1 否定式}\label{ux5426ux5b9aux5f0f}

设 \emph{p} 是任一命题,复合命题``非 \emph{p}''(或 ``\emph{p} 的否定'')
称为 \emph{p} 的否定式 (Negation),记作 \(\large \lnot p\),
``\(\lnot\)'' 为否定联结词。

\begin{longtable}[]{@{}ll@{}}
\toprule\noalign{}
p & \(\large \lnot p\) \\
\midrule\noalign{}
\endhead
\bottomrule\noalign{}
\endlastfoot
1 & 0 \\
0 & 1 \\
\end{longtable}

\subsubsection{4.2.2 合取}\label{ux5408ux53d6}

设 \emph{p}、\emph{Q} 是任两个命题,复合命题``\emph{p} 并且 \emph{Q} ''(
或 `` \emph{p} 和 \emph{Q} '') 称为 \emph{p} 与 \emph{Q} 的合取式
(\emph{Conjunction}),记作
\(\large p\land Q\),``\(\land\)''为合取联结词。

\begin{longtable}[]{@{}lll@{}}
\toprule\noalign{}
p & Q & \(\large p\land Q\) \\
\midrule\noalign{}
\endhead
\bottomrule\noalign{}
\endlastfoot
0 & 0 & 0 \\
0 & 1 & 0 \\
1 & 0 & 0 \\
1 & 1 & 1 \\
\end{longtable}

\subsubsection{4.2.3 析取}\label{ux6790ux53d6}

设 \emph{p、Q} 是任两个命题,复合命题``\emph{p} 或者 \emph{Q} ''称为 p
与 Q 的析取式 (Disjunction),记作 \(\large p\lor Q\),
``\(\lor\)''为析取联结词。

\begin{longtable}[]{@{}lll@{}}
\toprule\noalign{}
p & Q & \(\large p\lor Q\) \\
\midrule\noalign{}
\endhead
\bottomrule\noalign{}
\endlastfoot
0 & 0 & 0 \\
0 & 1 & 1 \\
1 & 0 & 1 \\
1 & 1 & 1 \\
\end{longtable}

\subsubsection{4.2.4 蕴涵}\label{ux8574ux6db5}

设 \emph{p、Q} 是任两个命题,复合命题``如果 \emph{p},则 \emph{Q} ''称为
p 与 Q 的蕴涵式 (implication),记作
\(\large p\to Q\),``\(\to\)''称为蕴涵联结词,p 称为蕴涵式的前件,Q
称为蕴涵式的后件。

\begin{longtable}[]{@{}lll@{}}
\toprule\noalign{}
p & Q & \(\large p\to Q\) \\
\midrule\noalign{}
\endhead
\bottomrule\noalign{}
\endlastfoot
0 & 0 & 1 \\
0 & 1 & 1 \\
1 & 0 & 0 \\
1 & 1 & 1 \\
\end{longtable}

\subsubsection{4.2.5 等价}\label{ux7b49ux4ef7}

设 p、Q 是任两个命题,复合命题``p 当且仅当 Q''称为 p 与 Q 的等价式
(Equivalence),记作
\(\large p\leftrightarrow Q\),``\(\leftrightarrow\)''称为等价联结词。

\begin{longtable}[]{@{}lll@{}}
\toprule\noalign{}
p & Q & \(\large p\leftrightarrow Q\) \\
\midrule\noalign{}
\endhead
\bottomrule\noalign{}
\endlastfoot
0 & 0 & 1 \\
0 & 1 & 0 \\
1 & 0 & 0 \\
1 & 1 & 1 \\
\end{longtable}

\subsection{4.3
自然语言与联结词互换}\label{ux81eaux7136ux8bedux8a00ux4e0eux8054ux7ed3ux8bcdux4e92ux6362}

\begin{enumerate}
\def\labelenumi{\arabic{enumi}.}
\tightlist
\item
  联结词``\(\lnot\)''是自然语言中的``非''、``不''和
  ``没有''等的逻辑抽象;
\item
  联结词``\(\land\)''是自然语言中的``并且''、``既\ldots{}
  又\ldots''、``但''、``和''等的逻辑抽象;
\item
  联结词``\(\lor\)''是自然语言中的``或''、``或者''
  等逻辑抽象;但``或''有``可兼或''、``不可兼或'' 二种,如:
\item
  联结词 `` \(\to\) '' 是自然语言中的 `` 如果 \ldots{} ,
  则\ldots'',``若\ldots,才能\ldots''、``除非\ldots,否则\ldots''
  等的逻辑抽象。主要描述方法有:

  \begin{enumerate}
  \def\labelenumii{\arabic{enumii}.}
  \tightlist
  \item
    因为 \emph{p} 所以 \emph{Q};
  \item
    只要 \emph{p} 就 \emph{Q};
  \item
    \emph{p} 仅当 \emph{Q};
  \item
    只有 \emph{Q},才 \emph{p};
  \item
    除非 \emph{Q},才 \emph{p};
  \item
    除非 \emph{Q},否则非 \emph{p};
  \item
    没有 \emph{Q},就没有 \emph{p}。
  \end{enumerate}
\item
  双条件联结词``⟷''是自然语言中的``充分必要条件''、``当且仅当''等的逻辑抽象;
\end{enumerate}

\subsection{4.4 公式的解释}\label{ux516cux5f0fux7684ux89e3ux91ca}

设 \(p_1、p_2、…、p_n\) 是出现在公式 \emph{G} 中的所有命题变元,指定
\(p_1、p_2、…、p_n\) 一组真值,则这组真值称为 \emph{G} 的一个解释,常记为
\emph{i}。

\begin{itemize}
\tightlist
\item
  如果公式 \emph{G} 在它的所有解释之下都为``真'',那么公式 \emph{G}
  称为永真公式 (重言式)。
\item
  如果公式 \emph{G} 在它的所有解释之下都为``假'',那么公式 \emph{G}
  称为永假公式 (矛盾式)。
\item
  如果公式 \emph{G} 不是永假的,那么公式 \emph{G} 称为可满足的。
\end{itemize}

设 \emph{G、H} 是公式,如果在任意解释 \emph{i} 下, \emph{G} 与 \emph{H}
的真值相同,则称公式 \emph{G、H} 是等价的,记作 \emph{G=H} 。

由于``=''不是一个联结词,而是一种关系,为此, 这种关系具有如下三个性质:

\begin{enumerate}
\def\labelenumi{\arabic{enumi}.}
\tightlist
\item
  自反性 \(G=G\);
\item
  对称性若\emph{G=H},则\emph{H=G};
\item
  传递性若\emph{G=H},\emph{H=S},则\emph{G=S}。
\end{enumerate}

\subsection{4.5
基本等价公式}\label{ux57faux672cux7b49ux4ef7ux516cux5f0f}

\begin{enumerate}
\def\labelenumi{\arabic{enumi})}
\tightlist
\item
  结合律
\end{enumerate}

\[
\huge
\begin{cases}
E_1:G\lor(H\lor S)=(G\lor H)\lor S \\
E_2: G\lor (H\lor S)=(G\lor H)\lor S
\end{cases}
\]

\begin{enumerate}
\def\labelenumi{\arabic{enumi})}
\setcounter{enumi}{1}
\tightlist
\item
  交换律
\end{enumerate}

\[
\huge
\begin{cases}
E_3:G\lor H=H\lor G\\
E_4:G\land H=H\land G
\end{cases}
\]

\begin{enumerate}
\def\labelenumi{\arabic{enumi})}
\setcounter{enumi}{2}
\tightlist
\item
  幂等律
\end{enumerate}

\[
\huge
\begin{cases}
E_5:G\lor G=G \\
E_6:G\land G=G
\end{cases}
\]

\begin{enumerate}
\def\labelenumi{\arabic{enumi})}
\setcounter{enumi}{4}
\tightlist
\item
  吸收律
\end{enumerate}

\[
\huge
\begin{cases}
E_7:G\lor (G\land H)= G \\
E_8:G\land (G\lor H)= G
\end{cases}
\]

\begin{enumerate}
\def\labelenumi{\arabic{enumi})}
\setcounter{enumi}{5}
\tightlist
\item
  分配律
\end{enumerate}

\[
\huge
\begin{cases}
E_9:G\lor (H\land S)=(G\lor H)\land (G\lor S)\\
E_{10}:G\land (H\lor S)=(G\land H)\lor (G\land S)
\end{cases}
\]

\begin{enumerate}
\def\labelenumi{\arabic{enumi})}
\setcounter{enumi}{6}
\tightlist
\item
  同一律
\end{enumerate}

\[
\huge
\begin{cases}
E_{11}:G\lor 0=G\\
E_{12}:G\land 1=G
\end{cases}
\]

\begin{enumerate}
\def\labelenumi{\arabic{enumi})}
\setcounter{enumi}{7}
\tightlist
\item
  零律
\end{enumerate}

\[
\huge
\begin{cases}
E_{13}:G\lor 1=1\\
E_{14}:G\land 0=0
\end{cases}
\]

\begin{enumerate}
\def\labelenumi{\arabic{enumi})}
\setcounter{enumi}{8}
\tightlist
\item
  排中律
\end{enumerate}

\[
\huge
E_{15}:G\lor \lnot G =1
\]

\begin{enumerate}
\def\labelenumi{\arabic{enumi})}
\setcounter{enumi}{9}
\tightlist
\item
  矛盾律
\end{enumerate}

\[
\huge
E_{16}:G\land \lnot G =0
\]

\begin{enumerate}
\def\labelenumi{\arabic{enumi})}
\setcounter{enumi}{10}
\tightlist
\item
  双重否定律
\end{enumerate}

\[
\huge
E_{17}:\lnot (\lnot G)=G
\]

\begin{enumerate}
\def\labelenumi{\arabic{enumi})}
\setcounter{enumi}{11}
\tightlist
\item
  \emph{De moRGan}定律
\end{enumerate}

\[
\huge
\begin{cases}
E_{18}:\lnot (G\lor H)=\lnot G\land \lnot H\\
E_{19}:\lnot (G\land H)=\lnot G\lor \lnot H
\end{cases}
\]

\begin{enumerate}
\def\labelenumi{\arabic{enumi})}
\setcounter{enumi}{12}
\tightlist
\item
  等价式
\end{enumerate}

\[
\huge
E_{20}: (G\leftrightarrow H)=(G\to H)\land (H\to G)
\]

\begin{enumerate}
\def\labelenumi{\arabic{enumi})}
\setcounter{enumi}{13}
\tightlist
\item
  蕴涵式
\end{enumerate}

\[
\huge
E_{21}:(G\to H)=(\lnot G\lor H)
\]

\begin{enumerate}
\def\labelenumi{\arabic{enumi})}
\setcounter{enumi}{14}
\tightlist
\item
  假言易位
\end{enumerate}

\[
\huge
E_{22}:G \to H=\lnot H\to \lnot G 
\]

\begin{enumerate}
\def\labelenumi{\arabic{enumi})}
\setcounter{enumi}{15}
\tightlist
\item
  等价否定等式
\end{enumerate}

\[
\huge
E_{23}:G \leftrightarrow H=\lnot G\leftrightarrow \lnot H 
\]

\begin{enumerate}
\def\labelenumi{\arabic{enumi})}
\setcounter{enumi}{16}
\tightlist
\item
  归谬论
\end{enumerate}

\[
\huge
E_{24}:(G \to H) \land (G\lnot H)=\lnot G 
\]

\subsection{4.6 范式}\label{ux8303ux5f0f}

\subsubsection{4.6.1
析取范式和合取范式}\label{ux6790ux53d6ux8303ux5f0fux548cux5408ux53d6ux8303ux5f0f}

\paragraph{4.6.1.1
文字,字句与短语}\label{ux6587ux5b57ux5b57ux53e5ux4e0eux77edux8bed}

\begin{enumerate}
\def\labelenumi{\arabic{enumi})}
\tightlist
\item
  命题变元或命题变元的否定称为文字
\item
  有限个文字的析取称为析取范式 (也称为子句)
\item
  有限个文字的合取称为合取范式 (也称为短语)
\item
  P 与¬ P 称为互补对。
\end{enumerate}

\begin{quote}
\textbf{一个命题变元或者其否定既可以是简单的子句,也可以是简单的短语。}
\end{quote}

\paragraph{4.6.1.2
析取范式和合取范式}\label{ux6790ux53d6ux8303ux5f0fux548cux5408ux53d6ux8303ux5f0f-1}

\begin{enumerate}
\def\labelenumi{\arabic{enumi})}
\tightlist
\item
  有限个短语的析取式称为析取范式
\end{enumerate}

\[
(P \land Q)\lor(P \land R) 
\]

\begin{enumerate}
\def\labelenumi{\arabic{enumi})}
\setcounter{enumi}{1}
\tightlist
\item
  有限个子句的合取式称为合取范式
\end{enumerate}

\[
(P \lor Q)\land(P \lor R)
\]

\begin{quote}
\textbf{特殊例子} 1. 单个的文字是子句、短语、析取范式,合取范式 2.
单个的子句是合取范式; 3. 单个的短语是析取范式; 4.
若单个的子句(短语)无最外层括号,则是合取范式(析取范式); 5.
析取范式、合取范式仅含联结词集\{¬,∧,∨\}; 6.
``¬''联结词仅出现在命题变元前。
\end{quote}

\subsubsection{4.6.2
主析取范式和主合取范式}\label{ux4e3bux6790ux53d6ux8303ux5f0fux548cux4e3bux5408ux53d6ux8303ux5f0f}

\paragraph{4.6.2.1
极小项和极大项}\label{ux6781ux5c0fux9879ux548cux6781ux5927ux9879}

\textbf{定义}:在含有 n 个命题变元 \(P_1、P_2、P_3、…、 P_n\)
的短语或子句中,若每个命题变元与其否定不同时存在,但二者之一恰好出现一次且仅一次,则称此短语或子句为关于
\(P_1、P_2、P_3、…、P_n\) 的一个极小项或极大项。

\begin{quote}
\textbf{极小项和极大项的性质} 1) 任意两个不同极小项的合取必为假; 2)
任意两个不同极大项的析取必为真; 3) 极大项的否定是极小项; 4)
极小项的否定是极大项; 5) 所有极小项的析取为永真公式; 6)
所有极大项的合取是永假公式。
\end{quote}

\paragraph{4.6.2.2
主析取范式和主合取范式}\label{ux4e3bux6790ux53d6ux8303ux5f0fux548cux4e3bux5408ux53d6ux8303ux5f0f-1}

\begin{enumerate}
\def\labelenumi{\arabic{enumi}.}
\tightlist
\item
  在给定的析取范式中,每一个短语都是极小项, 则称该范式为主析取范式
\end{enumerate}

\[
(P\land Q)\lor(\lnot P \land Q)
\]

\begin{enumerate}
\def\labelenumi{\arabic{enumi}.}
\setcounter{enumi}{1}
\tightlist
\item
  在给定的合取范式中,每一个子句都是极大项, 则称该范式为主合取范式
\end{enumerate}

\[
(P\lor Q)\land(\lnot P \lor Q)
\]

\begin{enumerate}
\def\labelenumi{\arabic{enumi}.}
\setcounter{enumi}{2}
\tightlist
\item
  如果一个主析取范式不包含任何极小项,则称该主析取范式为``空'';如果一个主合取范式不包含任何极大项,则称主合取范式为``空''
\end{enumerate}

\paragraph{4.6.2.3
主析取范式和主合取范式求法}\label{ux4e3bux6790ux53d6ux8303ux5f0fux548cux4e3bux5408ux53d6ux8303ux5f0fux6c42ux6cd5}

\subparagraph{1)求主析取范式}\label{ux6c42ux4e3bux6790ux53d6ux8303ux5f0f}

\begin{enumerate}
\def\labelenumi{\arabic{enumi}.}
\tightlist
\item
  找出真值表中其真值为真的行:
\item
  写出这些行所对应的极小项
\item
  将这些极小项进行析取即为该公式 G 的主析取范式。
\end{enumerate}

\subparagraph{2)求主合取范式}\label{ux6c42ux4e3bux5408ux53d6ux8303ux5f0f}

\begin{enumerate}
\def\labelenumi{\arabic{enumi}.}
\tightlist
\item
  找出真值表中其真值为假的行
\item
  写出这些行所对应的极大项
\item
  将这些极大项进行合取即为该公式 G 的主合取范式
\end{enumerate}

\subparagraph{3)两者之间的互换}\label{ux4e24ux8005ux4e4bux95f4ux7684ux4e92ux6362}

\[
\huge 
G=\lnot \lnot G = \lnot \left(\bigwedge\limits_{i=1}^{2^n-k} M_{j_i}\right) = \bigvee\limits_{i=1}^{2^n-k} \lnot M_{j_i} = \bigvee\limits_{i=1}^{2^n-k} m_{j_i}
\]

\paragraph{4.6.2.4
永真式和永假式}\label{ux6c38ux771fux5f0fux548cux6c38ux5047ux5f0f}

\begin{enumerate}
\def\labelenumi{\arabic{enumi})}
\tightlist
\item
  公式 G 为永真式当且仅当 G 的合取范式中每个简单的析取式 (子句)
  至少包含一个命题变元及其否定
\item
  公式 G 为永假式当且仅当 G 的析取范式中每个简单的合取式 (短语)
  至少包含一个命题变元及其否定
\item
  如果命题公式是永真公式当且仅当它的主析取范式包含所有的极小项,此时无主合取范式或者说主合取范式为``空''。
\item
  如果命题公式是永假公式当且仅当它的主合取范式包含所有的极大项,此时无主析取范式或者说主析取范式为``空''。
\item
  两个命题公式是相等的当且仅当它们对应的主析取范式之间相等,或者
  (可兼或) 它们对应的主合取范式之间相等。
\end{enumerate}

\section{第五章
谓词逻辑}\label{ux7b2cux4e94ux7ae0-ux8c13ux8bcdux903bux8f91}

当各命题之间的逻辑关系不是体现在原子命题之间,而是体现在构成原子命题的内部成分之间。对此,命题逻辑将无能为力,所以我们有了谓词逻辑。

\subsection{5.1
谓词逻辑中的基本概念}\label{ux8c13ux8bcdux903bux8f91ux4e2dux7684ux57faux672cux6982ux5ff5}

\subsubsection{定义 5.1.1}\label{ux5b9aux4e49-5.1.1}

在原子命题中,可以独立存在的客体 (句子中的主语、宾语等),称为个体词
(Individual)。而用以刻划客体的性质或客体之间的关系即是谓词 (Predicate)。

\begin{quote}
个体词的分类
\end{quote}

\begin{enumerate}
\def\labelenumi{\arabic{enumi}.}
\tightlist
\item
  表示具体的或特定的个体词称为个体常量 (Individual
  Constant),一般个体词常量用带或不带下标的小写英文字母
  \(a, b, c,…,a_1, b_1, c_1,…\) 等表示。
\item
  表示抽象的或泛指的个体词称为个体变量 (Individual
  Variable),一般用带或不带下标的小写英文字母
  \(x, y, z, …, x_1, y_1, z_1, …\) 等表示。
\end{enumerate}

\subsubsection{定义 5.1.2}\label{ux5b9aux4e49-5.1.2}

\begin{enumerate}
\def\labelenumi{\arabic{enumi})}
\tightlist
\item
  个体词的取值范围称为个体域 (或论域)(Individual Field),常用 \emph{D}
  表示;
\item
  而宇宙间的所有个体域聚集在一起所构成的个体域称为全总个体域 (Universal
  Individual Field)。
\end{enumerate}

\subsubsection{定义 5.1.3}\label{ux5b9aux4e49-5.1.3}

设 \emph{D} 为非空的个体域,定义在 \(D^n\)(表示 n 个个体都在个体域 D
上取值) 上取值于\{0,1\}上的 n 元函数,称为 n 元命题函数或 n 元谓词
(Propositional Function),记为 \(P(x_1, x_2, …, x_n)\)。此时,个体变量
\(x_1, x_2, …, x_n\) 的定义域都为 \emph{D},\(P(x_1, x_2, …, x_n)\)
的值域为\{0, 1}

\subsubsection{定义 5.1.4}\label{ux5b9aux4e49-5.1.4}

称 \((\large \forall x)\) 为全称量词(Universal Quantifier) ,
\((\large \exists x)\) 为存在量词( Existential Quantifier),其中的
\(x\) 称为作用变量(Function
Variable)。一般将其量词加在其谓词之前,记为
\((\large \forall x)F(x)\),\((\large \exists x)F(x)\)。此时,\(F(x)\)
称为全称量词和存在量词的辖域 (Scope)。

\subsubsection{定义 5.1.5}\label{ux5b9aux4e49-5.1.5}

统一个体域为全总个体域,而对每一个句子中个体变量的变化范围用一元特性谓词刻划之。这种特性谓词在加入到命题函数中时必定遵循如下原则:

\begin{enumerate}
\def\labelenumi{\arabic{enumi})}
\tightlist
\item
  对于全称量词 (``x),刻划其对应个体域的特性谓词作为蕴涵式之前件加入。
\item
  对于存在量词 (\$x),刻划其对应个体域的特性谓词作为合取式之合取项加入。
\end{enumerate}

\subsection{5.2
谓词的翻译原理}\label{ux8c13ux8bcdux7684ux7ffbux8bd1ux539fux7406}

\subsection{5.3
谓词的合式公式}\label{ux8c13ux8bcdux7684ux5408ux5f0fux516cux5f0f}

\subsection{5.4 谓词的标准型 -
范式}\label{ux8c13ux8bcdux7684ux6807ux51c6ux578b---ux8303ux5f0f}
