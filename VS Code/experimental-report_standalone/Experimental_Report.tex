\documentclass[supercite]{Experimental_Report}

\title{~~~~~~新生实践课~~~~~~}
\author{王李超}
\school{计算机科学与技术学院}
\classnum{CS启明2401}
\stunum{U202414887}
\instructor{范晔斌} % 李平、孙伟平、范晔斌、陈加忠
\date{2024年12月1日}

\usepackage{algorithm, multirow}
\usepackage{algpseudocode}
\usepackage{amsmath}
\usepackage{amsthm}
\usepackage{framed}
\usepackage{mathtools}
\usepackage{subcaption}
\usepackage{xltxtra} %提供了针对XeTeX的改进并且加入了XeTeX的LOGO, 自动调用xunicode宏包(提供Unicode字符宏)
\usepackage{bm}
\usepackage{tikz}
\usepackage{tikzscale}
\usepackage{pgfplots}
%\usepackage{enumerate}

\pgfplotsset{compat=1.16}

\newcommand{\cfig}[3]{
  \begin{figure}[htb]
    \centering
    \includegraphics[width=#2\textwidth]{images/#1.tikz}
    \caption{#3}
    \label{fig:#1}
  \end{figure}
}

\newcommand{\sfig}[3]{
  \begin{subfigure}[b]{#2\textwidth}
    \includegraphics[width=\textwidth]{images/#1.tikz}
    \caption{#3}
    \label{fig:#1}
  \end{subfigure}
}

\newcommand{\xfig}[3]{
  \begin{figure}[htb]
    \centering
    #3
    \caption{#2}
    \label{fig:#1}
  \end{figure}
}

\newcommand{\rfig}[1]{\autoref{fig:#1}}
\newcommand{\ralg}[1]{\autoref{alg:#1}}
\newcommand{\rthm}[1]{\autoref{thm:#1}}
\newcommand{\rlem}[1]{\autoref{lem:#1}}
\newcommand{\reqn}[1]{\autoref{eqn:#1}}
\newcommand{\rtbl}[1]{\autoref{tbl:#1}}

\algnewcommand\Null{\textsc{null }}
\algnewcommand\algorithmicinput{\textbf{Input:}}
\algnewcommand\Input{\item[\algorithmicinput]}
\algnewcommand\algorithmicoutput{\textbf{Output:}}
\algnewcommand\Output{\item[\algorithmicoutput]}
\algnewcommand\algorithmicbreak{\textbf{break}}
\algnewcommand\Break{\algorithmicbreak}
\algnewcommand\algorithmiccontinue{\textbf{continue}}
\algnewcommand\Continue{\algorithmiccontinue}
\algnewcommand{\LeftCom}[1]{\State $\triangleright$ #1}

\newtheorem{thm}{定理}[section]
\newtheorem{lem}{引理}[section]

\colorlet{shadecolor}{black!15}

\theoremstyle{definition}
\newtheorem{alg}{算法}[section]

\def\thmautorefname~#1\null{定理~#1~\null}
\def\lemautorefname~#1\null{引理~#1~\null}
\def\algautorefname~#1\null{算法~#1~\null}

\begin{document}

\maketitle

\clearpage

\pagenumbering{Roman}

\tableofcontents[level=2]
\clearpage

\pagenumbering{arabic}

\section{网页整体框架}

图\ref{fig1-1}这个网页的整体框架使用HTML与CSS实现,结构清晰简洁,包含头部、主体内容和底部三个主要部分。头部部分通过`<header>`标签呈现,显示网站标题“王李超的个人主页”,背景为黑色半透明,文字居中,视觉效果突出。主体内容区域由一个`<div>`容器构成,采用Flexbox布局,使内容水平居中对齐,容器内放置了五个链接盒子,分别通向“个人经历”“喜欢的音乐”“我的Steam”“Github主页”和“我的资源”页面。每个链接盒子有独特的背景图、文字说明,并添加了悬停时的动画效果,如放大、阴影变化及模糊背景图变清晰,提升交互体验。底部区域由`<footer>`标签定义,显示欢迎语和简单介绍,同样采用黑色半透明背景,使整体风格一致。整个页面的布局通过CSS控制,使网页在不同屏幕设备上自适应,从而提供良好的用户体验,整体风格现代、简洁而不失个性。
% \begin{enumerate}
% \renewcommand{\labelenumi}{\theenumi)}
% 	\item C++
% 	\item Java
% 	\item HTML
% \end{enumerate}

\begin{figure}[htb] % here top bottom
	\begin{center}
		\includegraphics[scale=0.80]{images/1-1.jpg}
		\caption{网页整体框架举例}
		\label{fig1-1}
	\end{center}
\end{figure}

\newpage

\section{主页设计}

1. 头部结构:头部区域使用`<header>`标签,显示网页标题“王李超的个人主页”,文字居中,采用白色字体和黑色透明背景。此设计增加了标题的视觉层次感,同时保持整体简洁大方。

2. 主体内容布局:主体区域通过一个`<div class="content">`容器承载,主要功能是展示五个链接盒子。每个盒子分别对应“个人经历”“喜欢的音乐”“我的Steam”“Github主页”和“我的资源”五个子页面,链接直观明了,方便导航。

3. 盒子设计:每个链接盒子都采用白色半透明背景,带有圆角和阴影效果。盒子内文字居中显示,使用粗体设计,整体呈现干净、现代的风格。盒子宽高比为黄金比例,使视觉效果更加和谐。

4. 动态交互效果:盒子设计中加入悬停时的动画效果,通过CSS实现。当鼠标悬停时,盒子会放大并增强阴影效果,同时背景图片从模糊变为清晰,使用户体验更加生动且有趣。

5. 布局实现:整个页面使用Flexbox布局,确保内容在不同设备和屏幕尺寸下都能保持良好的自适应效果。盒子排列居中,水平间距均匀,增强页面的对称性和美观度。

6. 底部结构:底部区域使用`<footer>`标签,展示欢迎信息和个人简介。采用黑色透明背景,与头部设计保持一致,文字居中排布,简洁而不失信息量。

7. 色彩与背景:页面整体以深色为主,通过透明背景与半透明盒子形成视觉对比。同时,页面背景为全屏图像,增加页面的层次感和视觉吸引力,使整体风格更具现代感。

8. 响应式设计:通过`<meta>`标签设置视口,使页面能自动适应各种设备屏幕尺寸,优化了移动设备的浏览体验,从而增强了网站的兼容性与实用性。

9. 设计思路总结:整个页面设计追求简洁、现代和互动性,突出用户体验。通过色彩、动画效果及布局优化,使用户在视觉和交互上都能感受到良好的体验。请见图\ref{fig2-1}。


\begin{figure}[htb]
	\begin{center}
		\includegraphics[scale=0.30]{images/2-1.png}
		\caption{主页举例}
		\label{fig2-1}
	\end{center}
\end{figure}

\newpage

\section{分页面设计}

\subsection{页面1 (个人信息页面设计)}

图\ref{fig3-1}该页面以展示“王李超的个人主页 - 个人资料”为主题,设计中遵循简洁、直观的原则,整体风格统一,易于用户浏览。页面头部使用\texttt{<header>}标签定义,标题“王李超的个人主页 - 个人资料”居中显示,采用黑色半透明背景与白色文字,形成简洁大方的视觉效果。通过适当的\texttt{padding},提升标题的可读性和整体页面的舒适感。主体部分通过\texttt{<div class="content">}容器包裹,采用Flexbox布局,使内容在垂直与水平方向居中对齐,确保在不同屏幕尺寸下依然保持居中状态。个人信息部分通过\texttt{<div class="project-details">}展示,每条信息单独使用\texttt{<div class="line">}包裹,增加白色字体与半透明背景,增强可读性和视觉层次感。每条信息块添加圆角、内边距及阴影效果,使用\texttt{box-shadow}提升立体感。此外,通过\texttt{white-space: nowrap}保证每段信息一行显示,使排版整洁。页面底部提供了跳转至Github的链接,用户点击即可访问相关资源。底部使用\texttt{<footer>}标签定义,背景风格与头部保持一致,采用黑色透明背景、白色文字,整体视觉风格和谐统一。通过\texttt{text-align: center}使欢迎信息与简要说明居中排列,进一步增强整体视觉的均衡感。为确保页面在各类设备上的适应性,通过\texttt{<meta>}视口设置,实现内容自适应布局。同时,Flexbox布局使各元素在不同屏幕尺寸下均保持居中,提高用户体验。文字内容设置最大宽度为视口宽度的100\%,防止内容溢出,提升排版稳定性。整体设计考虑了美观与实用性,确保在信息传递的同时,保持页面的简洁与高效。

\begin{figure}[htb]
	\begin{center}
		\includegraphics[scale=0.20]{images/3-1.png}
		\caption{主页举例}
		\label{fig3-1}
	\end{center}
\end{figure}

\begin{shaded*}
	\begin{alg}{网页源代码}
		\label{alg:1}
		\begin{algorithmic}
			\State \textbf{<!DOCTYPE html>}
			\State \textbf{<html lang="zh-CN">}
			\State \ \ \ \ \textbf{<head>}
			\State \ \ \ \ \ \ \ \ \ \textbf{<meta charset="UTF-8">}
			\State \ \ \ \ \ \ \ \ \ \textbf{<meta name="viewport" content="width=device-width, initial-scale=1.0">}
			\State \ \ \ \ \ \ \ \ \ \textbf{<title>王李超的个人主页 - 个人资料</title>}
			\State \ \ \ \ \ \ \ \ \ \textbf{<link rel="icon" href="../images/favicon.ico" type="image/x-icon">}
			\State \ \ \ \ \ \ \ \ \ \textbf{<style>}
			\State \ \ \ \ \ \ \ \ \ \ \ \textbf{body \{} \texttt{font-family: Arial, sans-serif;}
			\State \ \ \ \ \ \ \ \ \ \ \ \ \textbf{background-image: url('experience\_background.png');}
			\State \ \ \ \ \ \ \ \ \ \ \ \ \textbf{background-size: cover;}
			\State \ \ \ \ \ \ \ \ \ \ \ \ \textbf{background-position: center;}
			\State \ \ \ \ \ \ \ \ \ \ \ \ \textbf{background-attachment: fixed;}
			\State \ \ \ \ \ \ \ \ \ \ \ \ \textbf{color: hsl(0, 0\%, 0\%); }
			\State \ \ \ \ \ \ \ \ \ \ \ \ \textbf{display: flex;}
			\State \ \ \ \ \ \ \ \ \ \ \ \ \textbf{flex-direction: column;}
			\State \ \ \ \ \ \ \ \ \ \ \ \ \textbf{min-height: 100vh;}
			\State \ \ \ \ \ \ \ \ \ \ \ \textbf{<style>}
			\State \ \ \ \ \textbf{<body>}
			\State \ \ \ \ \ \ \ \ \textbf{<header>}
			\State \ \ \ \ \ \ \ \ \ \ \ \textbf{<h1>王李超的个人主页 - 个人资料</h1>}
			\State \ \ \ \ \ \ \ \ \textbf{</header>}
			\State \ \ \ \ \ \ \ \ \textbf{<div class="content">}
			\State \ \ \ \ \ \ \ \ \ \ \ \textbf{<div class="project-details">}
			\State \ \ \ \ \ \ \ \ \ \ \ \ \textbf{<div class="line">姓名:王李超</div>}
			\State \ \ \ \ \ \ \ \ \ \ \ \ \textbf{<div class="line">一看就是男的</div>}
			\State \ \ \ \ \ \ \ \ \ \ \ \ \textbf{<div class="line">广涉猎而不精</div>}
			\State \ \ \ \ \ \ \ \ \ \ \ \ \textbf{<div class="line">现在正在被学习折磨</div>}
			\State \ \ \ \ \ \ \ \ \ \ \ \ \textbf{<div class="line">暂时还没有项目在做</div>}
			\State \ \ \ \ \ \ \ \ \ \ \ \ \textbf{<div class="line">若想看我写的代码请移步<a href="web/github.html">Github</a></div>}
			\State \ \ \ \ \ \ \ \ \ \ \ \textbf{</div>}
			\State \ \ \ \ \ \ \ \ \textbf{</div>}
			\State \ \ \ \ \ \ \ \ \textbf{<footer>}
			\State \ \ \ \ \ \ \ \ \ \ \ \textbf{<h2>欢迎来到我的主页!</h2>}
			\State \ \ \ \ \ \ \ \ \ \ \ \textbf{<p>这里是王李超的个人主页,你可以通过上方的链接了解我的经历。</p>}
			\State \ \ \ \ \ \ \ \ \textbf{</footer>}
			\State \textbf{</body>}
			\State \textbf{</html>}
		\end{algorithmic}
	\end{alg}
	\end{shaded*}
	

\subsection{喜欢的音乐}

图\ref{fig3-2}这部分内容与主页面非常相似,不同的是修改了三个方框的大小 和模糊程度,使文字更加清晰可见。修改方框大小主要是为了适应专辑封面一比一的比例,以及让页面看起来更充实。

\begin{figure}[htb]
	\begin{center}
		\includegraphics[scale=0.30]{images/3-2.png}
		\caption{主页举例}
		\label{fig3-2}
	\end{center}
\end{figure}

以下是与主页面设计有所不同的部分
\begin{shaded*}
	\begin{alg}{不同的部分}
		\label{alg:1}
		\begin{algorithmic}
			\State \ \ \ \ \ \ \ \ \ \ \ \ \textbf{width: 400px;  height: 400px; }
			\State \ \ \ \ \ \ \ \ \ \ \ \ \textbf{opacity: 0.2; /* 默认虚化 */}
			\State \ \ \ \ \ \ \ \ \ \ \ \ \textbf{filter: blur(3px); /* 默认添加模糊效果 */}
		\end{algorithmic}
	\end{alg}
	\end{shaded*}

\subsection{我的steam主页}

图\ref{fig3-3}这个页面是直接保存的页面,没有做修改,主要是为了防止此页面直接跳转时无法正常显示页面背景没有做过多替换,在此就不放出源代码了。

\begin{figure}[htb]
	\begin{center}
		\includegraphics[scale=0.30]{images/3-3.png}
		\caption{主页举例}
		\label{fig3-3}
	\end{center}
\end{figure}

\subsection{我的GitHub主页}

图\ref{3-4}此页面与第一个页面功能有些许重复,但主要是为了提供一个对我现在所写的一个项目的介绍。
\newpage

\section{网页设计小结}

描述网页的设计和实现过程中遇到的问题及如何解决。描述网页的设计和实现过程中遇到的问题及如何解决。描述网页的设计和实现过程中遇到的问题及如何解决。描述网页的设计和实现过程中遇到的问题及如何解决。描述网页的设计和实现过程中遇到的问题及如何解决。描述网页的设计和实现过程中遇到的问题及如何解决。描述网页的设计和实现过程中遇到的问题及如何解决。描述网页的设计和实现过程中遇到的问题及如何解决。描述网页的设计和实现过程中遇到的问题及如何解决。

描述网页的设计和实现过程中遇到的问题及如何解决。描述网页的设计和实现过程中遇到的问题及如何解决。描述网页的设计和实现过程中遇到的问题及如何解决。描述网页的设计和实现过程中遇到的问题及如何解决。描述网页的设计和实现过程中遇到的问题及如何解决。描述网页的设计和实现过程中遇到的问题及如何解决。描述网页的设计和实现过程中遇到的问题及如何解决。描述网页的设计和实现过程中遇到的问题及如何解决。描述网页的设计和实现过程中遇到的问题及如何解决。

描述网页的设计和实现过程中遇到的问题及如何解决。描述网页的设计和实现过程中遇到的问题及如何解决。描述网页的设计和实现过程中遇到的问题及如何解决。描述网页的设计和实现过程中遇到的问题及如何解决。描述网页的设计和实现过程中遇到的问题及如何解决。描述网页的设计和实现过程中遇到的问题及如何解决。描述网页的设计和实现过程中遇到的问题及如何解决。描述网页的设计和实现过程中遇到的问题及如何解决。描述网页的设计和实现过程中遇到的问题及如何解决。

描述网页的设计和实现过程中遇到的问题及如何解决。描述网页的设计和实现过程中遇到的问题及如何解决。描述网页的设计和实现过程中遇到的问题及如何解决。描述网页的设计和实现过程中遇到的问题及如何解决。描述网页的设计和实现过程中遇到的问题及如何解决。描述网页的设计和实现过程中遇到的问题及如何解决。描述网页的设计和实现过程中遇到的问题及如何解决。描述网页的设计和实现过程中遇到的问题及如何解决。描述网页的设计和实现过程中遇到的问题及如何解决。

\newpage

\section{课程的收获和建议}

描述通过学习该专题,有何收获,有何建议,如某专题可适当减少讲授时间、某专题可适当增加讲授内容和时间等。描述通过学习该专题,有何收获,有何建议,如某专题可适当减少讲授时间、某专题可适当增加讲授内容和时间等。描述通过学习该专题,有何收获,有何建议,如某专题可适当减少讲授时间、某专题可适当增加讲授内容和时间等。描述通过学习该专题,有何收获,有何建议,如某专题可适当减少讲授时间、某专题可适当增加讲授内容和时间等。

\subsection{计算机基础知识}

描述通过学习计算机基础知识专题,有何收获,有何建议,如某专题可适当减少讲授时间、某专题可适当增加讲授内容和时间等。描述网页的设计和实现过程中遇到的问题及如何解决。描述网页的设计和实现过程中遇到的问题及如何解决。描述网页的设计和实现过程中遇到的问题及如何解决。描述网页的设计和实现过程中遇到的问题及如何解决。描述网页的设计和实现过程中遇到的问题及如何解决。描述网页的设计和实现过程中遇到的问题及如何解决。描述网页的设计和实现过程中遇到的问题及如何解决。描述网页的设计和实现过程中遇到的问题及如何解决。

\subsection{文档撰写工具LaTeX}

描述通过学习文档撰写工具LaTeX专题,有何收获,有何建议,如某专题可适当减少讲授时间、某专题可适当增加讲授内容和时间等。描述通过学习文档撰写工具LaTeX专题,有何收获,有何建议,如某专题可适当减少讲授时间、某专题可适当增加讲授内容和时间等。

\subsection{编程工具Python}

描述通过学习编程工具Python专题,有何收获,有何建议,如某专题可适当减少讲授时间、某专题可适当增加讲授内容和时间等。描述通过学习编程工具Python专题,有何收获,有何建议,如某专题可适当减少讲授时间、某专题可适当增加讲授内容和时间等。

\subsection{图像设计软件Photoshop}

描述通过学习计算机基础知识专题,有何收获,有何建议,如某专题可适当减少讲授时间、某专题可适当增加讲授内容和时间等。描述通过学习计算机基础知识专题,有何收获,有何建议,如某专题可适当减少讲授时间、某专题可适当增加讲授内容和时间等。

\subsection{版本管理软件Git}

描述通过学习图像设计软件Photoshop专题,有何收获,有何建议,如某专题可适当减少讲授时间、某专题可适当增加讲授内容和时间等。描述通过学习图像设计软件Photoshop专题,有何收获,有何建议,如某专题可适当减少讲授时间、某专题可适当增加讲授内容和时间等。

\subsection{网页制作Dreamweaver}

描述通过学习网页制作Dreamweaver专题,有何收获,有何建议,如某专题可适当减少讲授时间、某专题可适当增加讲授内容和时间等。描述通过学习网页制作Dreamweaver专题,有何收获,有何建议,如某专题可适当减少讲授时间、某专题可适当增加讲授内容和时间等。


\nocite{*} %% 作用是不对文献进行引用,但可以生成文献列表

%\bibliographystyle{HustGraduPaper}
%\bibliography{HustGraduPaper}

\end{document}
