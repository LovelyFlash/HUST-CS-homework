\documentclass[supercite]{Experimental_Report}

\title{~~~~~~新生实践课~~~~~~}
\author{王李超}
\school{计算机科学与技术学院}
\classnum{CS启明2401}
\stunum{U202414887}
\instructor{范晔斌} % 李平、孙伟平、范晔斌、陈加忠
\date{2021年11月11日}

\usepackage{algorithm, multirow}
\usepackage{algpseudocode}
\usepackage{amsmath}
\usepackage{amsthm}
\usepackage{framed}
\usepackage{mathtools}
\usepackage{subcaption}
\usepackage{xltxtra} %提供了针对XeTeX的改进并且加入了XeTeX的LOGO, 自动调用xunicode宏包(提供Unicode字符宏)
\usepackage{bm}
\usepackage{tikz}
\usepackage{tikzscale}
\usepackage{pgfplots}
%\usepackage{enumerate}

\pgfplotsset{compat=1.16}

\newcommand{\cfig}[3]{
  \begin{figure}[htb]
    \centering
    \includegraphics[width=#2\textwidth]{images/#1.tikz}
    \caption{#3}
    \label{fig:#1}
  \end{figure}
}

\newcommand{\sfig}[3]{
  \begin{subfigure}[b]{#2\textwidth}
    \includegraphics[width=\textwidth]{images/#1.tikz}
    \caption{#3}
    \label{fig:#1}
  \end{subfigure}
}

\newcommand{\xfig}[3]{
  \begin{figure}[htb]
    \centering
    #3
    \caption{#2}
    \label{fig:#1}
  \end{figure}
}

\newcommand{\rfig}[1]{\autoref{fig:#1}}
\newcommand{\ralg}[1]{\autoref{alg:#1}}
\newcommand{\rthm}[1]{\autoref{thm:#1}}
\newcommand{\rlem}[1]{\autoref{lem:#1}}
\newcommand{\reqn}[1]{\autoref{eqn:#1}}
\newcommand{\rtbl}[1]{\autoref{tbl:#1}}

\algnewcommand\Null{\textsc{null }}
\algnewcommand\algorithmicinput{\textbf{Input:}}
\algnewcommand\Input{\item[\algorithmicinput]}
\algnewcommand\algorithmicoutput{\textbf{Output:}}
\algnewcommand\Output{\item[\algorithmicoutput]}
\algnewcommand\algorithmicbreak{\textbf{break}}
\algnewcommand\Break{\algorithmicbreak}
\algnewcommand\algorithmiccontinue{\textbf{continue}}
\algnewcommand\Continue{\algorithmiccontinue}
\algnewcommand{\LeftCom}[1]{\State $\triangleright$ #1}

\newtheorem{thm}{定理}[section]
\newtheorem{lem}{引理}[section]

\colorlet{shadecolor}{black!15}

\theoremstyle{definition}
\newtheorem{alg}{算法}[section]

\def\thmautorefname~#1\null{定理~#1~\null}
\def\lemautorefname~#1\null{引理~#1~\null}
\def\algautorefname~#1\null{算法~#1~\null}

\begin{document}

\maketitle

\clearpage

\pagenumbering{Roman}

\tableofcontents[level=2]
\clearpage

\pagenumbering{arabic}

\section{网页整体框架}

图\ref{fig1-1}这个网页的整体框架使用HTML与CSS实现,结构清晰简洁,包含头部、主体内容和底部三个主要部分。头部部分通过`<header>`标签呈现,显示网站标题“王李超的个人主页”,背景为黑色半透明,文字居中,视觉效果突出。主体内容区域由一个`<div>`容器构成,采用Flexbox布局,使内容水平居中对齐,容器内放置了五个链接盒子,分别通向“个人经历”“喜欢的音乐”“我的Steam”“Github主页”和“我的资源”页面。每个链接盒子有独特的背景图、文字说明,并添加了悬停时的动画效果,如放大、阴影变化及模糊背景图变清晰,提升交互体验。底部区域由`<footer>`标签定义,显示欢迎语和简单介绍,同样采用黑色半透明背景,使整体风格一致。整个页面的布局通过CSS控制,使网页在不同屏幕设备上自适应,从而提供良好的用户体验,整体风格现代、简洁而不失个性。

% \begin{enumerate}
% \renewcommand{\labelenumi}{\theenumi)}
% 	\item C++
% 	\item Java
% 	\item HTML
% \end{enumerate}

\begin{figure}[htb] % here top bottom
	\begin{center}
		\includegraphics[scale=0.80]{images/1-1.jpg}
		\caption{网页整体框架举例}
		\label{fig1-1}
	\end{center}
\end{figure}

\newpage

\section{主页设计}

1. 头部结构:头部区域使用`<header>`标签,显示网页标题“王李超的个人主页”,文字居中,采用白色字体和黑色透明背景。此设计增加了标题的视觉层次感,同时保持整体简洁大方。

2. 主体内容布局:主体区域通过一个`<div class="content">`容器承载,主要功能是展示五个链接盒子。每个盒子分别对应“个人经历”“喜欢的音乐”“我的Steam”“Github主页”和“我的资源”五个子页面,链接直观明了,方便导航。

3. 盒子设计:每个链接盒子都采用白色半透明背景,带有圆角和阴影效果。盒子内文字居中显示,使用粗体设计,整体呈现干净、现代的风格。盒子宽高比为黄金比例,使视觉效果更加和谐。

4. 动态交互效果:盒子设计中加入悬停时的动画效果,通过CSS实现。当鼠标悬停时,盒子会放大并增强阴影效果,同时背景图片从模糊变为清晰,使用户体验更加生动且有趣。

5. 布局实现:整个页面使用Flexbox布局,确保内容在不同设备和屏幕尺寸下都能保持良好的自适应效果。盒子排列居中,水平间距均匀,增强页面的对称性和美观度。

6. 底部结构:底部区域使用`<footer>`标签,展示欢迎信息和个人简介。采用黑色透明背景,与头部设计保持一致,文字居中排布,简洁而不失信息量。

7. 色彩与背景:页面整体以深色为主,通过透明背景与半透明盒子形成视觉对比。同时,页面背景为全屏图像,增加页面的层次感和视觉吸引力,使整体风格更具现代感。

8. 响应式设计:通过`<meta>`标签设置视口,使页面能自动适应各种设备屏幕尺寸,优化了移动设备的浏览体验,从而增强了网站的兼容性与实用性。

9. 设计思路总结:整个页面设计追求简洁、现代和互动性,突出用户体验。通过色彩、动画效果及布局优化,使用户在视觉和交互上都能感受到良好的体验。请见图\ref{fig2-1}。

\begin{figure}[htb]
	\begin{center}
		\includegraphics[scale=0.30]{images/2-1.png}
		\caption{主页举例}
		\label{fig2-1}
	\end{center}
\end{figure}


\newpage

\section{分页面设计}

给出分页面截图,描述主要设计思路等。给出分页面截图,描述主要设计思路等。给出分页面截图,描述主要设计思路等。给出分页面截图,描述主要设计思路等。

\subsection{页面1 (每个页面以主要内容起标题名称即可)}

\section*{设计思路与框架}

\begin{enumerate}
    \item \textbf{头部设计}
    \begin{enumerate}
        \item 使用\texttt{<header>}标签定义,显示标题“王李超的个人主页 - 个人资料”。
        \item 背景采用黑色半透明效果,文字颜色为白色,突出简洁、大方的视觉效果。
        \item 文本居中显示,通过\texttt{padding}增加上下空间,提升阅读体验。
    \end{enumerate}

    \item \textbf{主体内容设计}
    \begin{enumerate}
        \item \textbf{内容容器结构}
        \begin{enumerate}
            \item 使用\texttt{<div class="content">}包裹整个内容部分,确保页面布局整体居中。
            \item 通过Flexbox布局实现内容在垂直和水平方向上的居中对齐。
        \end{enumerate}
        
        \item \textbf{个人信息展示}
        \begin{enumerate}
            \item 使用\texttt{<div class="project-details">}展示个人信息,包含多条个人简介。
            \item 每条信息通过\texttt{<div class="line">}标签呈现,采用白色字体和半透明背景,提升信息可读性。
        \end{enumerate}
        
        \item \textbf{细节设计}
        \begin{enumerate}
            \item \textbf{背景与文字}:背景使用深色风景图片,增强视觉层次感,文字颜色为白色,确保信息清晰。
            \item \textbf{样式效果}:每条信息增加圆角、阴影和内边距,通过\texttt{box-shadow}提升视觉层次感。
            \item \textbf{链接交互}:提供“Github”页面的超链接,用户可通过点击跳转至相关资源。
        \end{enumerate}
    \end{enumerate}

    \item \textbf{底部设计}
    \begin{enumerate}
        \item 使用\texttt{<footer>}标签定义底部区域,包含欢迎信息和简要说明。
        \item 背景和头部保持一致,黑色透明背景,白色文字,视觉风格统一。
        \item 通过\texttt{text-align: center}将文字居中显示,并适当调整字体大小和行距。
    \end{enumerate}

    \item \textbf{布局与响应式设计}
    \begin{enumerate}
        \item \textbf{自适应布局}
        \begin{enumerate}
            \item 使用\texttt{meta}视口设置,使页面在不同屏幕设备上自适应。
            \item Flexbox布局确保内容在各种屏幕尺寸下始终居中对齐,提升用户体验。
        \end{enumerate}
        
        \item \textbf{内容排版优化}
        \begin{enumerate}
            \item 使用\texttt{white-space: nowrap}保证每段信息一行显示,避免文字换行,提升信息整洁度。
            \item 设置最大宽度\texttt{max-width: 100\%},防止超出视口边界,提高排版的稳定性。
        \end{enumerate}
    \end{enumerate}
\end{enumerate}
如果实验报告中要用到算法伪代码,请参考算法\ref{alg:1},也可以参考算法\ref{alg:2}。如果实验报告中要用到算法伪代码,请参考算法\ref{alg:1},也可以参考算法\ref{alg:2}。如果实验报告中要用到算法伪代码,请参考算法\ref{alg:1},也可以参考算法\ref{alg:2}。如果实验报告中要用到算法伪代码,请参考算法\ref{alg:1},也可以参考算法\ref{alg:2}。

\begin{shaded*}\begin{alg}{一个复杂算法}
		\label{alg:1}
		\begin{algorithmic}
			\Input Two numbers $a$ and $b$
			\Output The sum of $a$ and $b$
			\Procedure{A-Plus-B}{$a, b$}
			\If $a = 0$
			\State \Return $b$
			\EndIf
			\State $res \gets 0$
			\While{$b \neq 0$}
			\State Increase $res$ by $1$
			\State $b \gets b - 1$
			\EndWhile
			\State \Return $res$
			\EndProcedure
		\end{algorithmic}
\end{alg}\end{shaded*}

\subsection{页面2 (每个页面以主要内容起标题名称即可)}

给出分页面截图,描述主要设计思路等。给出分页面截图,描述主要设计思路等。给出分页面截图,描述主要设计思路等。给出分页面截图,描述主要设计思路等。给出分页面截图,描述主要设计思路等。给出分页面截图,描述主要设计思路等。给出分页面截图,描述主要设计思路等。给出分页面截图,描述主要设计思路等。给出分页面截图,描述主要设计思路等。给出分页面截图,描述主要设计思路等。给出分页面截图,描述主要设计思路等。


如果实验报告中要用到算法伪代码,请参考算法\ref{alg:1},也可以参考算法\ref{alg:2}。如果实验报告中要用到算法伪代码,请参考算法\ref{alg:1},也可以参考算法\ref{alg:2}。如果实验报告中要用到算法伪代码,请参考算法\ref{alg:1},也可以参考算法\ref{alg:2}。如果实验报告中要用到算法伪代码,请参考算法\ref{alg:1},也可以参考算法\ref{alg:2}。

\begin{algorithm}[h] 
	\caption{一个更复杂算法}
	\begin{algorithmic}[1]
		\State Initialization: $I_{xy}$, $z_{f}=Zeros(128, 128)$; 
		\For{$0\leq n \textless N$}
		\State $i=\lfloor x_n \rfloor+64$, $j=\lfloor y_n \rfloor + 64$
		\If{$z_n<0$ and $|z_n|>|z_{f}(i,j)|$};
		\State $z_{f}(i,j)=z_n$;
		\EndIf
		\State $I_{xy}(i,j)=z_{f}(i,j)$;
		\EndFor 
	\end{algorithmic}\label{alg:2}
\end{algorithm}

\subsection{页面3 (每个页面以主要内容起标题名称即可)}

给出分页面截图,描述主要设计思路等。给出分页面截图,描述主要设计思路等。给出分页面截图,描述主要设计思路等。给出分页面截图,描述主要设计思路等。给出分页面截图,描述主要设计思路等。给出分页面截图,描述主要设计思路等。给出分页面截图,描述主要设计思路等。给出分页面截图,描述主要设计思路等。给出分页面截图,描述主要设计思路等。给出分页面截图,描述主要设计思路等。给出分页面截图,描述主要设计思路等。

\subsection{页面4 (每个页面以主要内容起标题名称即可)}

给出分页面截图,描述主要设计思路等。给出分页面截图,描述主要设计思路等。给出分页面截图,描述主要设计思路等。给出分页面截图,描述主要设计思路等。给出分页面截图,描述主要设计思路等。给出分页面截图,描述主要设计思路等。给出分页面截图,描述主要设计思路等。给出分页面截图,描述主要设计思路等。给出分页面截图,描述主要设计思路等。给出分页面截图,描述主要设计思路等。给出分页面截图,描述主要设计思路等。

\newpage

\section{网页设计小结}

描述网页的设计和实现过程中遇到的问题及如何解决。描述网页的设计和实现过程中遇到的问题及如何解决。描述网页的设计和实现过程中遇到的问题及如何解决。描述网页的设计和实现过程中遇到的问题及如何解决。描述网页的设计和实现过程中遇到的问题及如何解决。描述网页的设计和实现过程中遇到的问题及如何解决。描述网页的设计和实现过程中遇到的问题及如何解决。描述网页的设计和实现过程中遇到的问题及如何解决。描述网页的设计和实现过程中遇到的问题及如何解决。

描述网页的设计和实现过程中遇到的问题及如何解决。描述网页的设计和实现过程中遇到的问题及如何解决。描述网页的设计和实现过程中遇到的问题及如何解决。描述网页的设计和实现过程中遇到的问题及如何解决。描述网页的设计和实现过程中遇到的问题及如何解决。描述网页的设计和实现过程中遇到的问题及如何解决。描述网页的设计和实现过程中遇到的问题及如何解决。描述网页的设计和实现过程中遇到的问题及如何解决。描述网页的设计和实现过程中遇到的问题及如何解决。

描述网页的设计和实现过程中遇到的问题及如何解决。描述网页的设计和实现过程中遇到的问题及如何解决。描述网页的设计和实现过程中遇到的问题及如何解决。描述网页的设计和实现过程中遇到的问题及如何解决。描述网页的设计和实现过程中遇到的问题及如何解决。描述网页的设计和实现过程中遇到的问题及如何解决。描述网页的设计和实现过程中遇到的问题及如何解决。描述网页的设计和实现过程中遇到的问题及如何解决。描述网页的设计和实现过程中遇到的问题及如何解决。

描述网页的设计和实现过程中遇到的问题及如何解决。描述网页的设计和实现过程中遇到的问题及如何解决。描述网页的设计和实现过程中遇到的问题及如何解决。描述网页的设计和实现过程中遇到的问题及如何解决。描述网页的设计和实现过程中遇到的问题及如何解决。描述网页的设计和实现过程中遇到的问题及如何解决。描述网页的设计和实现过程中遇到的问题及如何解决。描述网页的设计和实现过程中遇到的问题及如何解决。描述网页的设计和实现过程中遇到的问题及如何解决。

\newpage

\section{课程的收获和建议}

描述通过学习该专题,有何收获,有何建议,如某专题可适当减少讲授时间、某专题可适当增加讲授内容和时间等。描述通过学习该专题,有何收获,有何建议,如某专题可适当减少讲授时间、某专题可适当增加讲授内容和时间等。描述通过学习该专题,有何收获,有何建议,如某专题可适当减少讲授时间、某专题可适当增加讲授内容和时间等。描述通过学习该专题,有何收获,有何建议,如某专题可适当减少讲授时间、某专题可适当增加讲授内容和时间等。

\subsection{计算机基础知识}

描述通过学习计算机基础知识专题,有何收获,有何建议,如某专题可适当减少讲授时间、某专题可适当增加讲授内容和时间等。描述网页的设计和实现过程中遇到的问题及如何解决。描述网页的设计和实现过程中遇到的问题及如何解决。描述网页的设计和实现过程中遇到的问题及如何解决。描述网页的设计和实现过程中遇到的问题及如何解决。描述网页的设计和实现过程中遇到的问题及如何解决。描述网页的设计和实现过程中遇到的问题及如何解决。描述网页的设计和实现过程中遇到的问题及如何解决。描述网页的设计和实现过程中遇到的问题及如何解决。

\subsection{文档撰写工具LaTeX}

描述通过学习文档撰写工具LaTeX专题,有何收获,有何建议,如某专题可适当减少讲授时间、某专题可适当增加讲授内容和时间等。描述通过学习文档撰写工具LaTeX专题,有何收获,有何建议,如某专题可适当减少讲授时间、某专题可适当增加讲授内容和时间等。

\subsection{编程工具Python}

描述通过学习编程工具Python专题,有何收获,有何建议,如某专题可适当减少讲授时间、某专题可适当增加讲授内容和时间等。描述通过学习编程工具Python专题,有何收获,有何建议,如某专题可适当减少讲授时间、某专题可适当增加讲授内容和时间等。

\subsection{图像设计软件Photoshop}

描述通过学习计算机基础知识专题,有何收获,有何建议,如某专题可适当减少讲授时间、某专题可适当增加讲授内容和时间等。描述通过学习计算机基础知识专题,有何收获,有何建议,如某专题可适当减少讲授时间、某专题可适当增加讲授内容和时间等。

\subsection{版本管理软件Git}

描述通过学习图像设计软件Photoshop专题,有何收获,有何建议,如某专题可适当减少讲授时间、某专题可适当增加讲授内容和时间等。描述通过学习图像设计软件Photoshop专题,有何收获,有何建议,如某专题可适当减少讲授时间、某专题可适当增加讲授内容和时间等。

\subsection{网页制作Dreamweaver}

描述通过学习网页制作Dreamweaver专题,有何收获,有何建议,如某专题可适当减少讲授时间、某专题可适当增加讲授内容和时间等。描述通过学习网页制作Dreamweaver专题,有何收获,有何建议,如某专题可适当减少讲授时间、某专题可适当增加讲授内容和时间等。


\nocite{*} %% 作用是不对文献进行引用,但可以生成文献列表

%\bibliographystyle{HustGraduPaper}
%\bibliography{HustGraduPaper}

\end{document}
